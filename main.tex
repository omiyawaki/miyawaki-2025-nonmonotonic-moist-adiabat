%% Version 7.1, 1 September 2021
%
%%%%%%%%%%%%%%%%%%%%%%%%%%%%%%%%%%%%%%%%%%%%%%%%%%%%%%%%%%%%%%%%%%%%%%
% amspaperV6.tex --  LaTeX-based instructional template paper for submissions to the 
% American Meteorological Society
%
%%%%%%%%%%%%%%%%%%%%%%%%%%%%%%%%%%%%%%%%%%%%%%%%%%%%%%%%%%%%%%%%%%%%%
% PREAMBLE
%%%%%%%%%%%%%%%%%%%%%%%%%%%%%%%%%%%%%%%%%%%%%%%%%%%%%%%%%%%%%%%%%%%%%

%% Start with one of the following:
% 1.5-SPACED VERSION FOR SUBMISSION TO THE AMS
\documentclass[draft]{ametsocV6.1}

% TWO-COLUMN JOURNAL PAGE LAYOUT---FOR AUTHOR USE ONLY
% \documentclass[twocol]{ametsocV6.1}

%%%%%%%%%%%%%%%%%%%%%%%%%%%%%%%%

%%% To be entered by author:

%% May use \\ to break lines in title:

\title{The non-monotonicity of moist adiabatic warming}

%% Enter authors' names and affiliations as you see in the examples below.
%
%% Use \correspondingauthor{} and \thanks{} (\thanks command to be used for affiliations footnotes, 
%% such as current affiliation, additional affiliation, deceased, co-first authors, etc.)
%% immediately following the appropriate author.
%
%% Note that the \correspondingauthor{} command is NECESSARY.
%% The \thanks{} commands are OPTIONAL.
%
%% Enter affiliations within the \affiliation{} field. Use \aff{#} to indicate the affiliation letter at both the
%% affiliation and at each author's name. Use \\ to insert line breaks to place each affiliation on its own line.

\authors{Osamu Miyawaki\aff{a}\correspondingauthor{Osamu Miyawaki, miyawako@union.edu}}

\affiliation{\aff{a}{Department of Geosciences, Union College, Schenectady New York, USA}}

%%%%%%%%%%%%%%%%%%%%%%%%%%%%%%%%%%%%%%%%%%%%%%%%%%%%%%%%%%%%%%%%%%%%%
% ABSTRACT
%
% Enter your abstract here
% Abstracts should not exceed 250 words in length!
%

\abstract{}

\begin{document}


%% Necessary!
\maketitle

%%%%%%%%%%%%%%%%%%%%%%%%%%%%%%%%%%%%%%%%%%%%%%%%%%%%%%%%%%%%%%%%%%%%%
% MAIN BODY OF PAPER
%%%%%%%%%%%%%%%%%%%%%%%%%%%%%%%%%%%%%%%%%%%%%%%%%%%%%%%%%%%%%%%%%%%%%
%
\section{Introduction}

The Clausius-Clapeyron relation implies the potential for a warmer atmosphere to hold more water vapor \citep{emanuel1994}. This principle is the basis for the positive water vapor feedback \citep{held2000a} and various scaling theories in response to warming including extreme precipitation, Hadley cell edge, jet stream position, tropopause height, and CAPE \citep{oGorman2015, shaw2016b, romps2016}. 

In the tropics, convection couples the surface with the free troposphere. Because the timescale of radiative cooling is slower (order of days) compared to convection (order of hours), the tropical atmosphere is to first order in a state of quasi-equilibrium where the climatological free-tropospheric temperature follows a convectively neutral temperature profile set by the surface temperature and humidity \cite{arakawa1974}. Although processes like convective entrainment influence the details of this coupling \citep{miyawaki2020, keil2021}, moist adiabatic adjustment serves as a useful first-order approximation \citep{held1993}. The top-heavy warming profile predicted by moist adiabatic adjustment (Fig.~\ref{fig:fig-1}b) is a robust feature in climate models and observations, despite historical challenges in observational records \citep{vallis2015, santer2005}.

The top-heavy warming profile predicted by the moist adiabat is important because it increases dry static stability. The spatial variations in the dry static stability response influences the structure of tropical convergence zones because horizontal free-tropospheric gradients, while weak, exist \citep{neelin1987, bao2022}. This structure also defines the tropical lapse rate feedback, a key negative feedback for global climate sensitivity \citep{hansen1984}. The lapse rate feedback partially cancels the water vapor feedback and scales in tandem because amplified warming in the upper troposphere is a consequence of enhanced latent heat release \citep{held2012}. In a moist adiabatic atmosphere that is saturated at the surface, total latent heat release $L_v (q_s^*-q_\mathrm{top}^*)$ where $L_v$ is the latent heat of vaporization, $q_s^*$ is surface saturation specific humidity, and $q_\mathrm{top}^*$ is the cloud top saturation specific humidity. $q_\mathrm{top}^*\to0$ as $T\to0$ in a moist adiabatic atmosphere because the moist adiabat does not predict a stratosphere\footnote{\cite{romps2016} presents a more realistic analysis of how $q_\mathrm{top}^*$ varies with $T_s$ by assuming a fixed tropopause temperature $= 200$~K \citep[see ][for evidence supporting this assumption]{hartmann2002, seeley2019}. $q_\mathrm{top}^*$ exhibits super Clausius-Clapeyron scaling driven by decreasing cloud top pressure with warming.}. Thus we expect total latent heat release in a moist adiabatic atmosphere to scale as $q_s^*$, which increases monotonically with surface temperature as expected from the Clausius-Clapeyron relation (Fig.~\ref{fig:fig-1}a).

Given the monotonic increase in surface specific humidity with surface temperature, one might expect moist adiabatic warming to also increase monotonically with the initial surface temperature at all levels. However, it is a non-monotonic function of initial surface temperature at fixed pressure levels (Fig.~\ref{fig:fig-1}c, see Appendix~A for details on the calculation). This non-monotonicity arises in height coordinates (Fig.~\ref{fig:fig-a1}), with or without latent heat of fusion (see Appendix~B and Fig.~\ref{fig:fig-b1}), and across different empirical formula for saturation vapor pressure (see Appendix~C and Fig.~\ref{fig:fig-c1}). While previous work has acknowledged the existence of this non-monotonicity \citep{byrne2013, levine2016}, an explanation for the non-monotonicity in moist adiabatic warming does not yet exist in the literature.

This raises the question: What physical mechanism drives this non-monotonic warming? Here we derive a thermodynamic explanation for the origin of non-monotonicity in moist adiabatic warming and its cascading effects on buoyancy and vertical velocity.

\begin{figure}[htbp]
 \centering
 \includegraphics[width=0.4\textwidth]{fig-1.png}\\
\caption{(a) Surface saturation specific humidity increases monotonically with surface temperature. (b) Vertical profiles of moist adiabatic warming to a 4~K surface warming, plotted against pressure, for $T_s = $ 280, 290, 300, 310, and 320~K. Warming decreases with initial surface temperature at lower levels while it increases with initial surface temperature at higher levels. (c) Moist adiabatic warming varies non-monotonically with initial surface temperature at all levels, e.g. at 500, 400, 300, and 200~hPa. Moist adiabatic warming peaks at warmer initial surface temperatures at higher levels.}
\label{fig:fig-1}
\end{figure}

\section{Theory of Non-Monotonic Warming}
We start by defining the moist adiabatic temperature profile in pressure coordinates $T(p)$ in terms of the moist adiabatic lapse rate $\Gamma_m = dT/dp$:
\begin{equation}
T(p) = T_s + \int_{p_s}^{p} \Gamma_m \, dp' \label{eq:temp_profile_pressure}
\end{equation}
where $T_s$ is surface temperature. We assume the atmosphere is saturated from the surface. The difference between a perturbed and baseline state ($\Delta$) then follows as
\begin{equation}
\Delta T(p) = \Delta T_s + \int_{p_s}^{p} \Delta\Gamma_m \, dp' \label{eq:delta_t_pressure}
\end{equation}
For a small perturbation, $\Delta \Gamma_m$ can be approximated using a first-order Taylor expansion: $\Delta\Gamma_m \approx \frac{d\Gamma_m}{dT_s}\Delta T_s$. Substituting this into Eq.~(\ref{eq:delta_t_pressure}) gives:
\begin{equation}
\Delta T(p) \approx \Delta T_s + \left(\int_{p_s}^{p} \frac{d\Gamma_m}{dT_s}dp'\right)\Delta T_s \label{eq:delta_t_taylor_pressure}
\end{equation}
Thus the non-monotonicity in moist adiabatic warming is encoded into $d\Gamma_m/dT_s$, the sensitivity of the moist adiabatic lapse rate to surface temperature. Indeed, $d\Gamma_m/dT_s$ is non-monotonic with respect to temperature (dashed line shows the local minima of $d\Gamma_m/dT_s$ in Fig.~\ref{fig:fig-2}a). Note that $d\Gamma_m/dT_s$ is mostly negative in the troposphere (Fig.~\ref{fig:fig-2}b). This is consistent with amplified warming aloft because the integral in Eq.~(\ref{eq:delta_t_pressure} is from high to low pressure, which introduces a negative sign.

$\Gamma_m$ is a function of local state variables $\Gamma_m(T, p)$. Thus to make progress in understanding $d\Gamma_m/dT_s$, we must rewrite $d\Gamma_m/dT_s$ in terms of derivatives with respect to the local state variables $(T, p)$. To do this we first use the chain rule: 
\begin{equation}
\frac{d\Gamma_m}{dT_s} = \left(\frac{\partial\Gamma_m}{\partial T}\right)_p \cdot \frac{dT}{dT_s} + \left(\frac{\partial\Gamma_m}{\partial p}\right)_T \cdot \frac{dp}{dT_s} \label{eq:chain_rule_start}
\end{equation}
The second term $\frac{dp}{dT_s}=0$ because pressure is the vertical coordinate and is an independent variable. Recognizing that by definition $\Gamma_m = \frac{dT}{dp}$,
\begin{equation}
    \frac{d}{dp}\left(\frac{dT}{dT_s}\right) = \left(\frac{\partial\Gamma_m}{\partial T}\right)_p \cdot \frac{dT}{dT_s} 
    \label{eq:ode}
\end{equation}
This is an ordinary differential equation for $\frac{dT}{dT_s}$ as a function of pressure. The solution with the boundary condition $\frac{dT}{dT_s}(p_s) = 1$, is:
\begin{equation}
    \frac{dT}{dT_s} = \exp\left(\int_{p_s}^{p} \left(\frac{\partial\Gamma_m}{\partial T}\right)_p dp'\right)
    \label{eq:ode-solution}
\end{equation}
Substituting Eq.~(\ref{eq:ode-solution}) back into Eq.~(\ref{eq:chain_rule_start}) gives:
\begin{equation}
\frac{d\Gamma_m}{dT_s} = \left(\frac{\partial\Gamma_m}{\partial T}\right)_p \cdot \exp\left(\int_{p_s}^{p} \left(\frac{\partial\Gamma_m}{\partial T}\right)_{p'} dp'\right) \label{eq:total_sensitivity}
\end{equation}
where $(\partial\Gamma_m/\partial T)_p$ is the moist adiabatic lapse rate sensitivity to local temperature $T$ at pressure level $p$. The integral describes how a small surface temperature perturbation $dT_s$ influences $\Gamma_m$ through the sum of all $\Gamma_m$ changes that occur below $p$.

The non-monotonicity can arise from either 1) $(\partial\Gamma_m/\partial T)_p$ being non-monotonic and the integral acting to amplify it or 2) $(\partial\Gamma_m/\partial T)_p$ being monotonic but sign changes in $(\partial\Gamma_m/\partial T)_p$ leads to the integral being non-monotonic. Numerical solutions show that $(\partial\Gamma_m/\partial T)_p$ is non-monotonic (dash-dot line shows the local minima of $d\Gamma_m/dT$ in Fig.~\ref{fig:fig-2}c), which is further amplified by the integral term (Fig.~\ref{fig:fig-2}d).

Why is $(\partial\Gamma_m/\partial T)_p$ non-monotonic with $T$? To understand this we solve for $\Gamma_m$ from the first law of thermodynamics for adiabatic, non-precipitating, and reversible ascent of a saturated air parcel, which is equivalent to the conservation of saturation moist static energy:
\begin{equation}
c_{p} dT - \alpha dp + L_v dq^* = 0 \label{eq:mse_pressure}
\end{equation}
where $c_{p}$ is the specific heat capacity of air at constant pressure, $\alpha$ is specific volume, $L_v$ is the latent heat of vaporization, and $q^*$ is the saturation specific humidity. We assume 1) $c_p \approx c_{pd}$, neglecting the role of water of all phases on the specific heat capacity and 2) $\alpha \approx \alpha_d = R_d T/p$, neglecting the virtual effect of water vapor on density. 

Use the chain rule to expand $dq^*$:
\begin{equation}
dq^* = \left(\frac{\partial q^*}{\partial T}\right)_p dT + \left(\frac{\partial q^*}{\partial p}\right)_T dp \label{eq:dqs_expansion}
\end{equation}

Substituting Eq.~(\ref{eq:dqs_expansion}) into Eq.~(\ref{eq:mse_pressure}) and rearranging gives
\begin{equation}
\left(c_{pd} + L_v\left(\frac{\partial q^*}{\partial T}\right)_p \right)dT = \left(\alpha_d - L_v\left(\frac{\partial q^*}{\partial p}\right)_T\right)dp \label{eq:rearranged}
\end{equation}
We can derive closed-form expressions for the $q^*$ derivatives using the Clausius-Clapeyron relation and Dalton's Law. These $q^*$ derivatives describe the role of phase equilibrium shifts in $q^*$ with $T$ and $p$ on the effective heat capacity and specific volume of the air parcel, respectively:
\begin{align}
c_L &\equiv L_v\left(\frac{\partial q^*}{\partial T}\right)_p \approx \frac{L_v^2 q^*}{R_v T^2}
\label{eq:c_L} \\
\alpha_L &\equiv -L_v\left(\frac{\partial q^*}{\partial p}\right)_T \approx \frac{L_v q^*}{p}
\label{eq:alpha_L}
\end{align}
where the approximation arises from assuming saturation vapor pressure $e^* \ll p$.

$c_L$ can be thought of as a latent heat capacity, representing the enhanced thermal inertia due to the fact that latent heating buffers some of the cooling from expansion. Thus $c_L$ acts to increase the heat capacity of the air parcel such that it has an effective heat capacity $c_{pd} + c_L$.

$\alpha_L$ can be thought of as a latent specific volume, representing the enhanced expansion of air with ascent due to the fact that lower pressure shifts the phase equilibrium of water to favor the vapor phase over liquid. Thus $\alpha_L$ acts to increase the volume of air such that it has an effective specific volume $\alpha_d + \alpha_L$.

Now solving for the moist adiabatic lapse rate $\Gamma_m = dT/dp$:
\begin{align}
\Gamma_m = \frac{dT}{dp} &= \frac{\alpha_d +\alpha_L}{c_{pd} + c_L} \label{eq:gamma_m_ratio} \\
&= \Gamma_d \cdot \frac{1+\frac{\alpha_L}{\alpha_d}}{1+\frac{c_L}{c_{pd}}} \label{eq:gamma_m_factored}
\end{align}
where $\Gamma_d = \alpha_d / c_{pd}$ is the dry adiabatic lapse rate in pressure coordinates and the two non-dimensional terms represent the fractional increase in effective heat capacity and specific volume due to phase equilibrium changes:
\begin{align}
\tilde{c} &= \frac{c_L}{c_{pd}} = \frac{L_v^2 q^*}{c_{pd} R_v T^2} \label{eq:c_ratio} \\
\tilde{\alpha} &= \frac{\alpha_L}{\alpha_d} = \frac{L_v q^*}{R_d T} = \frac{R_v c_{pd}T}{R_dL_v}\tilde{c} = k\tilde{c} \label{eq:alpha_ratio}
\end{align}
Substituting Eq.~(\ref{eq:c_ratio}) and Eq.~(\ref{eq:alpha_ratio}) into Eq.~(\ref{eq:gamma_m_factored}) gives:
\begin{equation}
\Gamma_m = \Gamma_d \cdot \frac{1 + k\tilde{c}}{1 + \tilde{c}} \label{eq:gamma_m_tilde}
\end{equation}

For typical values in Earth's atmosphere ($R_v=461$ J kg$^{-1}$ K$^{-1}$, $R_d=287$ J kg$^{-1}$ K$^{-1}$, $c_{pd}=1005$ J kg$^{-1}$ K$^{-1}$, $L_v=2.5\times10^6$ J kg$^{-1}$, and $T \in [200, 320]$ K), the factor $k=\frac{R_v c_{pd}T}{R_dL_v}\in [0.13, 0.21]$. Thus $k$ is a weak function of temperature and is a quasi-constant of order $10^{-1}$. In contrast, $\tilde{c}$ scales exponentially with temperature (through $q^*$) and varies from $\tilde{c}(200\text{ K})\sim 10^{-4}$ to $\tilde{c}(320\text{ K})\sim 10^{1}$. Thus the temperature sensitivity of $\Gamma_m$ is controlled by $\tilde{c}$. Because $\Gamma_m$ is bounded between $\Gamma_d$ (dry limit, $\tilde{c} \to 0$) and $k\Gamma_d$ (moist limit, $\tilde{c} \to \infty$), the magnitude of $\partial\Gamma_m/\partial T$ must peak at some intermediate $\tilde{c}$ else $\Gamma_m$ would be unbounded.

Where does the magnitude of $\partial\Gamma_m/\partial T$ reach its peak value? To solve this we use the quotient rule on Eq.~(\ref{eq:gamma_m_ratio}):
\begin{equation}
\frac{\partial\Gamma_m}{\partial T} = \underbrace{\frac{1}{c_{pd} + c_L}\frac{\partial(\alpha_d + \alpha_L)}{\partial T}}_{\text{latent volume sensitivity}} + \underbrace{\left(-\frac{\alpha_d + \alpha_L}{(c_{pd} + c_L)^2}\frac{\partial c_L}{\partial T}\right)}_{\text{latent heat capacity sensitivity}} \label{eq:decomposition}
\end{equation}
The latent volume sensitivity varies monotonically with $T_s$ (Fig.~\ref{fig:fig-3}a, c). The latent heat capacity sensitivity varies non-monotonically with $T_s$ (Fig.~\ref{fig:fig-3}b, d). Thus we further decompose the latent heat capacity sensitivity to probe its origin:
\begin{equation}
-\frac{\alpha_d + \alpha_L}{(c_{pd} + c_L)^2}\frac{\partial c_L}{\partial T} = -\frac{1}{p} \cdot \left(1 + \tilde{\alpha}\right) \cdot \frac{R_d}{c_{pd}}\frac{\partial\log{c_L}}{\partial \log{T}} \cdot f_d \cdot f_L \label{eq:term_b_intermediate}
\end{equation}
where
\begin{equation}
f_d \equiv c_{d}/(c_{pd} + c_L) \label{eq:f_d}
\end{equation}
\begin{equation}
f_L \equiv c_{L}/(c_{pd} + c_L) \label{eq:f_L}
\end{equation}
and $f_d + f_L = 1$. $f_d$ and $f_L$ represent the dry and latent fractions of effective heat capacity. 

Eq.~(\ref{eq:term_b_intermediate}) shows the latent heat capacity sensitivity is a product of four terms that vary monotonically with $T$. $\tilde{\alpha}=L_v q^* / (\alpha_d p)$ scales exponentially with $T$ through $q^*$ (red line in Fig.~\ref{fig:fig-4}a). The fractional change in latent heat capacity to a fractional change in temperature $\partial\log{c_L}/\partial\log{T} = L_v / (R_v T) - 2$ so it weakly decreases with $T$ (blue line in Fig.~\ref{fig:fig-4}a). The product of these two terms is weakly non-monotonic in $T$ with a local minimum where $\tilde{\alpha} \approx R_vT/L_v$ (white line in Fig.~\ref{fig:fig-4}b). At low $T$, $\tilde{\alpha}$ is small so the product is dominated by the decrease in $\partial\log{c_L}/\partial\log{T}$. At high $T$, $\tilde{\alpha}$ is large so the product is dominated by the exponential increase in $\tilde{\alpha}$ . However, the non-monotonicity of these two terms are not the source of the peak in the magnitude of $\partial\Gamma_m/\partial T$, which requires a local maximum, not a minimum.

The dry fraction of effective heat capacity $f_d=c_{pd}/(c_{pd}+c_L)$ logistically decreases with $T$ because $c_{pd}$ is a constant while latent heat capacity $c_L$ increases exponentially with $T$ through $q^*$ (blue line in Fig.~\ref{fig:fig-4}c). The latent fraction of effective heat capacity $f_L = c_L / (c_{pd}+c_L)$ logistically increases with $T$ (red line in Fig.~\ref{fig:fig-4}c). The product $f_d\cdot f_L$ is maximized when $f_d=f_L$, or $c_L = c_{pd}$ (black line in Fig.~\ref{fig:fig-4}d).

What is the physical intuition behind the peak at $c_L = c_{pd}$? Recall that $c_L$ quantifies the enhancement of effective heat capacity due to latent heat of condensation offsetting adiabatic cooling. The $q^*$ derivative in $c_L$ requires two ingredients: 1) cooling from expansion and 2) water vapor. $f_d$ and $f_L$ represent the fractional availability of the two ingredients. At low $T$, condensation is limited by the availability of water vapor (red line in Fig.~\ref{fig:fig-4}c). At high $T$ condensation is limited by adiabatic cooling (blue line in Fig.~\ref{fig:fig-4}c). The peak in latent heat capacity sensitivity corresponds to where the availability of cooling and water vapor are equally limiting (black line in Fig.~\ref{fig:fig-4}c). Thus the non-monotonicity in $\partial\Gamma_m/\partial T$ and moist adiabatic warming arises from the competition between the two limiting factors of condensation.

How well does the condition $c_L = c_{pd}$ capture the actual peak in $\partial\Gamma_m/\partial T$? The theory slightly overpredicts the $T_s$ where the magnitude of $\partial\Gamma_m/\partial T$ peaks (compare solid and dash-dot lines in Fig.~\ref{fig:fig-5}). This error is due to the weak non-monotonicity in the product $(1+\tilde{\alpha})R_d/c_{pd}\partial\log(c_L)/\partial\log(T)$ which decreases with pressure (Fig.~\ref{fig:fig-4}b). The error maximizes at the surface where the theory predicts a peak $T_s$ that is 1.6 K warmer than the true peak $T_s$. 

The error in $T_s$ predicted by the theory and the true peak of $\Gamma_m / d T_s$ grows with height because the integral term in Eq.~(\ref{eq:total_sensitivity}) amplifies the error in $\partial\Gamma_m / \partial T$ at each level below. This error maximizes at 420 hPa where $c_L = c_{pd}$ predicts a peak $T_s$ that is 2.0 K warmer than the true peak $T_s$ (compare solid and dashed lines in Fig.~\ref{fig:fig-5}). This error is further compounded for $T_s$ corresponding to the peak of moist adiabatic warming $\Delta T$ (Eq.~\ref{eq:delta_t_taylor_pressure}), leading to a maximum error of 6.6 K at 420 hPa (compare solid and dotted lines in Fig.~\ref{fig:fig-5}). Thus the condition $c_L = c_{pd}$ is a useful first-order estimate of $T_s$ where moist adiabatic warming peaks. Importantly the theory successfully captures the shift to warmer peak $T_s$ with height, which is due to the fact that temperature decreases with height and thus the transition from the vapor limited to cooling limited regime occurs at a warmer surface temperature with height.

\begin{figure}[htbp]
 \centering
 \includegraphics[width=\textwidth]{fig-2.png}\\
 \caption{(a) The sensitivity of the moist adiabatic lapse rate to surface temperature, $d\Gamma_m/d T_s$, varies non-monotonically with surface temperature. (b) $d\Gamma_m/d T_s$ has a local minimum across surface temperature at all pressure levels, e.g. across 500, 400, 300, and 200~hPa. A minimum in $d\Gamma_m/d T_s$ corresponds to a maximum in moist adiabatic warming (Fig.~\ref{fig:fig-1}b) because the integral bounds in Eq.~\ref{eq:delta_t_taylor_pressure} decreases from $p_s$ to $p$, which introduces a negative sign. The local minimum shifts toward warmer with surface temperature at higher levels. (c) The sensitivity of the moist adiabatic lapse rate to the local temperature at pressure $p$, $\partial\Gamma_m/\partial T$, also varies non-monotonically with surface temperature. (d) The integral term in Eq.~(\ref{eq:total_sensitivity}) amplifies the non-monotonicity of $\partial\Gamma_m/\partial T$. (a) is the product of (c) and (d).}\label{fig:fig-2}
\end{figure}

\begin{figure}[htbp]
 \centering
 \includegraphics[width=\textwidth]{fig-3.png}\\
 \caption{The moist adiabatic lapse rate sensivity to local temperature $T$, $\partial \Gamma_d/\partial T$ (Fig.~\ref{fig:fig-2}c), is decomposed into contributions from (a) the latent volume sensivity and (b) the latent heat capacity sensitivity following Eq.~(\ref{eq:decomposition}). (c) The latent volume sensitivity monotonically increases increases with local temperature $T$ across all pressure levels, e.g. across 500, 400, 300, and 200~hPa. (d) The latent heat capacity sensitivity has a local minimum that shifts toward warmer surface temperature at higher levels, consistent with the behavior of $d \Gamma_d/dT_s$ (Fig.~\ref{fig:fig-2}b).}\label{fig:fig-3}
\end{figure}

\begin{figure}[htbp]
 \centering
 \includegraphics[width=\textwidth]{fig-4.png}\\
 \caption{The latent volume sensitivity is decomposed into a product of four terms (Eq.~\ref{eq:term_b_intermediate}) that varies monotonically with local temperature $T$, where local means at pressure $p$. (a) The latent volume ratio $\tilde{\alpha}$ increases exponentially with local temperature (red line) while the fractional change in latent heat capacity $c_L$ to a fractional change in local temperature $T$ decreases linearly with $T$ (blue line). The product of the two is weakly non-monotonic with $T$ where the product has a local minimum (purple line). (b) The local minimum approximately occurs where $\tilde{\alpha}= R_v T / L_v$ (white line). (c) The latent heat capacity fraction $f_L$ increases logistically with local temperature $T$ (red line) while the dry heat capacity fraction $f_d$ decreases logistically with $T$ (blue line). The product of the two is non-monotonic with $T$ where the product has a local maximum (purple line). (d) The local maximum occurs where $c_L=c_{pd}$ (black line).}\label{fig:fig-4}
\end{figure}


\section{Implications of non-monotonicity in moist adiabatic warming on convection}
The non-monotonic warming of a moist adiabat has implications for the dynamics of convection. For example, \cite{romps2016} showed that parcel buoyancy is a non-monotonic function of surface temperature. Specifically the criterion where $B$ peaks is $\beta = 2c_{pd}$ where
\begin{equation}
\beta = c_{pd} + L_v\frac{\partial q^*}{\partial T} = c_{pd} + c_L
\end{equation}
Thus the criterion that maximizes $B$ is equivalent to where moist adiabatic warming peaks, $c_{pd} = c_L$. Below, we show this is true if the entrainment parameter $a = PE \epsilon / g$\footnote{$PE$ is precipitation efficiency, $\epsilon$ is the fractional entrainment rate, and $g$ is gravitational acceleration. See \cite{romps2016} for the derivation of the entraining plume model.} is small and derive a more general criterion that maximizes $B$. 

Buoyancy $B$ is the normalized virtual temperature (or equivalently, density) difference between the rising parcel $T_{v,p}$ and the environment $T_{v,e}$. Here we neglect the virtual effects of water and we use standard temperature:
\begin{equation}
B\approx\frac{g}{T_e}(T_p-T_e) \label{eq:buoyancy_def}
\end{equation}
As before, we express temperature profiles in terms of $T_s$ and the integral of their respective lapse rates. We assume the parcel follows a moist adiabatic lapse rate, $\Gamma_m$, while the environment is neutrally buoyant with respect to an entraining lapse rate, $\Gamma_e$ following the zero-buoyancy plume model introduced by \cite{singh2013}.
\begin{align}
T_p&=T_s+\int_{p_s}^p \Gamma_m(p') \, dp' \label{eq:Tp_profile} \\
T_e&=T_s+\int_{p_s}^p \Gamma_e(p') \, dp' \label{eq:Te_profile}
\end{align}
Substituting Eq.~(\ref{eq:Tp_profile}) and (\ref{eq:Te_profile}) into the definition of buoyancy Eq.~(\ref{eq:buoyancy_def}) yields:
\begin{equation}
B\approx\frac{g}{T_e}\int_{p_s}^p \delta \Gamma \, dp' \label{eq:buoyancy_integral}
\end{equation}
where $\delta\Gamma = \Gamma_e - \Gamma_m$. We use the same entraining plume model as in \cite{romps2016} but express the lapse rate in pressure coordinates:
\begin{equation}
\Gamma_e = \Gamma_d \cdot \frac{(1+a)\alpha_d + \alpha_L}{(1+a)c_{pd}+c_L} \label{eq:gamma_e}
\end{equation}
Substituting Eq.~(\ref{eq:gamma_m_ratio}) and (\ref{eq:gamma_e}) into Eq.~(\ref{eq:buoyancy_integral}) and simplifying gives:
\begin{equation}
    B = \frac{g}{T_e}\int_{p_s}^p \Gamma_d \cdot \frac{a(1-k)\tilde{c}}{(1+a+\tilde{c})(1+\tilde{c})} \, dp' \label{eq:buoyancy_final}
\end{equation}
If we assume that $a$ does not vary with $T_s$, $T_e$ increases monotonically with $T_s$ at all $p$. The origin of the non-monotonicity of $B$ must be in the integrand, $\delta \Gamma$. $B$ depends on $T$ primarily through $\tilde{c}$, which scales exponentially with $T$ through $q^*$, whereas $\Gamma_d$ and $k$ are linear functions of $T$. In the limit of $\tilde{c} \to 0$ (cold and dry), $\delta\Gamma$ scales as $\tilde{c}$, which increases with $T$. In the limit of $\tilde{c} \to \infty$ (warm and humid), $\delta\Gamma$ scales as $\tilde{c}^{-1}$, which decreases with increasing $T$. Thus the integrand maximizes at some intermediate $\tilde{c}$.

To solve for the condition that maximizes buoyancy we solve for the $\tilde{c}$ derivative of the integrand $\delta \Gamma$ in Eq.~(\ref{eq:buoyancy_final}) and set it to zero:
\begin{equation}
    \frac{d}{d \tilde{c}}\left(\Gamma_d \cdot \frac{a(1-k)\tilde{c}}{(1+a+\tilde{c})(1+\tilde{c})}\right) = 0 \label{eq:buoyancy_derivative}
\end{equation}
If we assume that $a$, $k$, and $\Gamma_d$ do not vary with $T$, the solution to Eq.~(\ref{eq:buoyancy_derivative}) is
\begin{equation}
    \tilde{c}_\text{peak}=\sqrt{1+a} \label{eq:buoyancy_quadratic}
\end{equation}
Thus the condition that maximizes buoyancy is $c_L = \sqrt{1+a} c_{pd}$. In the limit of weak entrainment $a \to 0$, this reduces to $c_L = c_{pd}$. In the presence of entrainment, buoyancy peaks at a higher $c_L$ and thus higher $T_s$ all else equal. Entrainment reduces the latent heat released by the cooling parcel given the same $q^*$ so it shifts the critical point that separates the vapor limited and cooling limited regimes toward higher $q^*$.

How important is the factor $\sqrt{1+a}$? For an entrainment rate representative of Earth's current climate $a=0.2$, the difference in peak $T_s$ that corresponds to $c_L=c_{pd}$ and $c_L=\sqrt{1+a}c_{pd}$ are $< 1.49$ K (compare red and solid black line in Fig.~\ref{fig:fig-6}a). This difference decreases with height and becomes negligibly small around the tropopause (e.g., 0.33 K at $p=100$ hPa), which explains why the criteria $c_L = c_{pd}$ works well for explaining the non-monotonicity of CAPE \citep{romps2016}. However, for stronger entrainment rates and for understanding the non-monotonicity of buoyancy in the lower troposphere the factor $\sqrt{1+a}$ can be important (e.g., 4.38 K for $a=0.7$ at the surface; compare red and solid black line in Fig.~\ref{fig:fig-6}b).


How well do these criteria capture the $T_s$ that maximizes buoyancy across the troposphere? We will first focus on $\delta \Gamma$, i.e. the integrand in Eq.~(\ref{eq:buoyancy_integral}). For $a=0.2$ both criteria capture the $T_s$ that corresponds to the peak in $\delta \Gamma$ well ($<1.39$ K for $c_L=\sqrt{1+a}c_{pd}$, $< 2.87$ K for $c_L=c_{pd}$, compare red and solid black line to dashed line in Fig.~\ref{fig:fig-6}a). The small error arises even for the $c_L=\sqrt{1+a}c_{pd}$ criterion because $\Gamma_d(1-k)$ is weakly non-monotonic with $T$ ($\Gamma_d$ increases with $T$ and $(1-k)$ decreases with $T$), which we ignored in order to analytically solve Eq.~(\ref{eq:buoyancy_derivative}). This error is amplified as we integrate $\delta \Gamma$ to obtain buoyancy Eq.~(\ref{eq:buoyancy_integral}) because the errors in the location of peak $\delta \Gamma$ from each level below accumulates for the location of peak $B$ compare red and solid black line to dotted line in Fig.~\ref{fig:fig-6}a).

For a higher entrainment parameter $a=0.7$ the importance of the factor $\sqrt{1+a}$ becomes clear. The error in $T_s$ that corresponds to the peak in $\delta \Gamma$ is $<3.39$ K for the $c_L=\sqrt{1+a}c_{pd}$ criterion compared to $<5.83$ K for the $c_L=c_{pd}$ criterion (compare red and solid black line to dashed line in Fig.~\ref{fig:fig-6}b). The error in $T_s$ that corresponds to the peak in buoyancy is surprisingly lower for the $c_L=c_{pd}$ criterion ($<3.37$ K) compared to the $c_L=\sqrt{1+a}c_{pd}$ criterion ($<4.66$ K, compare red and solid black lines to dotted black line in Fig.~\ref{fig:fig-6}b). This is because $c_L=c_{pd}$ underpredicts $T_s$ for peak $B$ in the lower troposphere, which offsets the growth of the larger error in peak $\delta \Gamma$ (compare solid black and dotted lines in Fig.~\ref{fig:fig-6}b). The criteria $c_L=c_{pd}$ predicts the $T_s$ of peak buoyancy better than $c_L=c_{pd}\sqrt{1+a}$ in some cases because of a cancelation of errors rather than for the right physical reason. For example the criteria $c_L=c_{pd}$ predicts no shift in $T_s$ that maximizes $B$ to perturbations in $a$ while the criterion $c_L=\sqrt{1+a}c_{pd}$ captures the shift in peak $\delta \Gamma$ and $B$ toward warmer $T_s$ with increasing entrainment (Fig.~\ref{fig:fig-6}c).

This non-monotonic behavior of buoyancy extends to the strength of the convective updraft. We model the updraft's specific kinetic energy, $\frac{1}{2}w^2$, using Eq.~(1) from \cite{delgenio2007}:
\begin{equation}
\frac{d}{dz}\left(\frac{1}{2}w^2\right)=a'B(z)-(1+b')\epsilon(z)w^2 \label{eq:momentum}
\end{equation}
where $a'$ and $b'$ are dimensionless constants. We use $a'=1/6$ and $b'=2/3$ following \cite{delgenio2007}. $\epsilon(z)$ is the fractional entrainment rate, which is calculated following Eq.~(3) in \cite{romps2016} with entrainment parameter $a=0.2$ and precipitation efficiency $PE=0.35$. Since $w(z)$ is determined by the integral of the net force, which includes buoyancy, we expect the non-monotonic dependence on $T_s$ extends to the vertical velocity profile as well.

Numerically integrating Eq.~(\ref{eq:momentum}) confirms this expectation. The resulting vertical velocity varies non-monotonically with $T_s$ (Fig.~\ref{fig:fig-7}b). This leads w(z) becoming more top-heavy with warming, i.e. w decreases in the lower troposphere and increases in the upper troposphere (Fig.~\ref{fig:fig-7}a).

Are these findings relevant to Earth's atmosphere, where convection is not strictly moist adiabatic and vertical velocity is subject to details and constraints not considered here such as cloud microphysics and radiative cooling? \cite{singh2015} shows the 99.99th percentile vertical velocity in CM1 exhibits a response to surface temperature variations that is qualitatively consistent with Eq.~(\ref{eq:momentum}) (compare our Fig.~\ref{fig:fig-7}a to their Fig.~2). However, in their model the decrease in vertical velocity with warming is confined to the lower troposphere below $p\approx900$~hPa whereas Eq.~(\ref{eq:momentum}) predicts a decrease in vertical velocity with warming over a deeper layer of the atmosphere (e.g., below $z\approx11$~km at 300~K). Thus to better understand the applicability and robustness of our results we analyzed vertical velocity in 9 convective-resolving models simulating radiative convective equilibrium in a 100 km x 100 km domain from the RCEMIP project \citep{wing2018}. We look at the mean vertical velocity profiles for $w$ exceeding the 99.9th percentile at each height level over the last 25 days of each simulation. The 99.9th percentile corresponds to the fastest 1000 samples of $w$ per level per model. We focus on the strongest convective updrafts because the convective core of the strongest updrafts are most conducive to follow parcel profiles that are close to a moist adiabat \citep{riehl1958}.

The vertical velocity profiles from the RCEMIP simulations show diverse $w_{>99.9}$ responses to variations in surface temperature (295, 300, and 305 K, see Fig.~\ref{fig:fig-8}). Some models exhibit a clear top-heavy shift in $w_{>99.9}$ with warming (e.g., CM1, DAM, UCLA-CRM, UKMO, WRF) accompanied by a decrease in $w_{>99.9}$ in the lower troposphere that is qualitatively consistent with the moist adiabatic theory (Fig.~\ref{fig:fig-7}a). SAM shows a top-heavy shift in $w_{>99.9}$ without a clear decrease in $w_{>99.9}$ in the lower troposphere. In the remaining models the $w_{>99.9}$ response exhibits non-monotonicity with $T_s$ but the peak $w_{>99.9}$ does not necessarily increase. For example DALES and SCALE predict a non-monotonic response in $w_{>99.9}$ with $T_s$ at $z\approx8$ km but the peak $w_{>99.9}$ weakens from $T_s=300$ to 305 K. MesoNH also predicts a decrease in peak $w_{>99.9}$ from $T_s=300$ K to 305 K but predicts a non-monotonic response in $w_{>99.9}$ with $T_s$ at $z\approx3$ km, much lower than in DALES and SCALE. The diversity in responses likely arises from differences in model details such as parameterization schemes for cloud microphysics, radiative transfer, and turbulence in addition to emergent behavior such as convective organization that influence convective dynamics beyond the basic thermodynamic processes considered here. Nonetheless, the presence of non-monotonicity and a shift toward increasingly top-heavy updraft velocity profiles in several models suggest that the underlying mechanism controlling non-monotonicity in moist adiabatic warming may be playing a role in shaping the sensitivity of updraft velocity to surface temperature in models that explictly resolve convective storms.

\begin{figure}[htbp]
 \centering
 \includegraphics[width=\textwidth]{fig-5.png}\\
 \caption{Surface temperature $T_s$ corresponding to the criteria $c_L=c_{pd}$ (solid), the minimum of the moist adiabatic lapse rate sensitivity to local temperature $\partial \Gamma_m/\partial T$ (dash dot), the minimum of the moist adiabatic lapse rate sensitivity to surface temperature $d \Gamma_m/dT_s$ (dashed), and the maximum of moist adiabatic warming $\Delta T$ (dotted). The theory most accurately captures the $T_s$ corresponding to the minimum of $\partial \Gamma_m /\partial T$. The discrepancy between the theory and the $T_s$ corresponding to the minimum of $d\Gamma_m/dT_s$ and $\Delta T$ are larger because the error at pressure $p$ is the accumulation of errors at levels below $p$ (see Eq.~\ref{eq:total_sensitivity} and \ref{eq:delta_t_taylor_pressure}).}\label{fig:fig-5}
\end{figure}

\begin{figure}[htbp]
 \centering
 \includegraphics[width=\textwidth]{fig-6.png}\\
 \caption{Surface temperature $T_s$ corresponding to the criterion $c_L=c_{pd}$ (solid black), the criterion $c_L=c_{pd}\sqrt{1+a}$ (red), the maximum of buoyancy $B$ (dotted), and the minimum of the difference between an entraining lapse rate and moist adiabatic lapse rate $\delta \Gamma = \Gamma_e - \Gamma_m$ (dashed) for the entrainment parameter (a) $a=0.2$ and (b) $a=0.7$. (c) The criterion $c_L=c_{pd}\sqrt{1+a}$ captures the $a$ dependence of $\delta \Gamma$ and $B$ extrema evaluated at pressure $p=500$~hPa. In comparison the criterion $c_L=c_{pd}$ is not sensitive to the entrainment parameter $a$ (vertical black line).}\label{fig:fig-6}
\end{figure}

\begin{figure}[htbp]
 \centering
 \includegraphics[width=\textwidth]{fig-7.png}\\
 \caption{(a) Vertical profiles of vertical velocity, calculated by numerically integrating Eq.~(\ref{eq:momentum}) in height using buoyancy $B$ from Eq.~(\ref{eq:buoyancy_def}). Vertical velocity decreases with surface temperature at lower levels while it increases with surface temperature at higher levels. (b) Vertical velocity varies non-monotonically with surface temperature at all levels, e.g. at 5, 10, 15, and 20~km. Vertical velocity peaks at warmer surface temperatures at higher levels consistent with the behavior of buoyancy (Fig.~\ref{fig:fig-6}a) and moist adiabatic warming (Fig.~\ref{fig:fig-5}).}\label{fig:fig-7}
\end{figure}

\clearpage

\begin{figure}[htbp]
 \centering
 \includegraphics[width=\textwidth]{fig-8.png}\\
 \caption{Updraft velocity from 9 cloud resolving models (CM1, DALES, DAM, MesoNH, SAM-CRM, SCALE, UCLA-CRM, UKMO-CASIM, and WRF) that participated in RCEMIP \citep{wing2018}. The simulations are on a 100~km $\times$ 100~km periodic domain for uniform sea surface temperatures set to 295 (blue), 300 (black), and 305~K (red). Updraft velocity at each level is the mean of vertical velocities $w$ that exceed the 99.9th percentile ($w_{>99.9}$).}\label{fig:fig-8}
\end{figure}


\section{Summary and Discussion}

Moist adiabatic warming varies non-monotonically with respect to surface temperature. The non-monotonicity arises because of a competition between two limiting factors of condensation: availability of water vapor and adiabatic cooling. At low surface temperatures, condensation is limited by the availability of water vapor and thus moist adiabatic warming scales like Clausius-Clapeyron while at high surface temperatures condensation is limited by adiabatic cooling. The surface temperature where moist adiabatic warming peaks approximately follows $c_L = c_{pd}$, where $c_L = L_v \partial q^* / \partial T$. The non-monotonicity of moist adiabatic warming propagates to buoyancy because buoyancy scales as the difference between an entraining lapse rate and the moist adiabatic lapse rate. The surface temperature where buoyancy peaks approximately follows $c_L = c_{pd} \sqrt{1+a}$, where $a$ is the entrainment parameter as defined in \cite{romps2016}. Finally the non-monotonicity of buoyancy propagates to the vertical velocity profile of convective updrafts. Cloud resolving models simulating radiative convective equilibrium exhibit diverse but qualitatively consistent responses in strong convective updrafts to surface temperature changes.

\cite{seeley2016} first noted the importance of the $c_L=c_{pd}$ criteria to explain why buoyancy profiles are top heavy. Buoyancy maximizes where the saturation moist static energy difference between the environment and parcel ($\delta h^*$) are expressed as a temperature difference (sensible enthalpy difference, $c_{pd}\delta T$) rather than a humidity difference (latent enthalpy difference, $L_v\delta q^*$). Thus the ratio $\tilde{c}= \ c_L / c_{pd} = L_v \delta q^* / (c_{pd} \delta T)$ quantifies the transition in the layers of the atmosphere where $\delta h^*$ is expressed largely in terms of $L_v \delta q^*$ (lower troposphere, where $\tilde{c}\gg1$) and in terms of $c_{pd}\delta T$ (upper troposphere, where $\tilde{c}\ll1$). \cite{seeley2015a} showed that buoyancy at a fixed level varies non-monotonically with surface temperature (see their Fig.~2a) and \cite{romps2016} explained the non-monotonicity of buoyancy to explain the non-monotonicity of CAPE with surface temperature. Following the same reasoning as in \cite{seeley2016}, \cite{romps2016} shows that buoyancy and thus CAPE peaks where $c_L = c_{pd}$. Here we find the same criterion $c_L = c_{pd}$ also explains the non-monotonicity in moist adiabatic warming. We also expand on \cite{romps2016} by showing that a more general criterion for the $T_s$ corresponding to the peak in buoyancy is $c_L = c_{pd}\sqrt{1+a}$, which reduces to \cite{romps2016}'s criterion in the limit of zero entrainment. The factor $\sqrt{1+a}$ is insignificant in Earth's current climate (e.g. for $a=0.2$, $\sqrt{1+a}=1.09$) thus \cite{romps2016}'s criterion works well for understanding the non-monotonicity of CAPE. However, the factor $\sqrt{1+a}$ becomes important for generalizing the theory to stronger entrainment rates and for understanding the non-monotonicity of buoyancy in the lower troposphere.

The non-monotonicity of moist adiabatic warming may have additional implications for climate, such as the organization of convection and the large-scale circulation response to warming. For example, \cite{shaw2025}  explain the mechanism behind the $2\%$~K$^{-1}$ scaling of the mean and extreme upper-level wind response to warming by assuming the atmosphere is moist adiabatic. The non-monotonicity of moist adiabatic warming would drive a non-monotonic change in the meridional and zonal temperature gradients at fixed height or pressure levels. This could serve as a thermodynamically driven hypothesis for the potential of non-monotonicities to emerge in response surface warming of dynamics such as the jet stream wind, extratropical cyclones, and mean overturning circulations.

%%%%%%%%%%%%%%%%%%%%%%%%%%%%%%%%%%%%%%%%%%%%%%%%%%%%%%%%%%%%%%%%%%%%%
% ACKNOWLEDGMENTS
%%%%%%%%%%%%%%%%%%%%%%%%%%%%%%%%%%%%%%%%%%%%%%%%%%%%%%%%%%%%%%%%%%%%%
\acknowledgments
I thank Andrew Williams, Jiawei Bao, Jonah Bloch-Johnson, Martin Singh, Stephen Po-Chedley, Nadir Jeevanjee, and two anonymous reviewers for helpful discussions and feedback on the manuscript.

%%%%%%%%%%%%%%%%%%%%%%%%%%%%%%%%%%%%%%%%%%%%%%%%%%%%%%%%%%%%%%%%%%%%%
% DATA AVAILABILITY STATEMENT
%%%%%%%%%%%%%%%%%%%%%%%%%%%%%%%%%%%%%%%%%%%%%%%%%%%%%%%%%%%%%%%%%%%%%
% 
%
\datastatement
All scripts used for analysis and plots in this paper are available at \url{https://github.com/omiyawaki/miyawaki-2025-nonmonotonic-moist-adiabat}. They will also be archived on Zenodo upon publication.


%%%%%%%%%%%%%%%%%%%%%%%%%%%%%%%%%%%%%%%%%%%%%%%%%%%%%%%%%%%%%%%%%%%%%
% APPENDIXES
%%%%%%%%%%%%%%%%%%%%%%%%%%%%%%%%%%%%%%%%%%%%%%%%%%%%%%%%%%%%%%%%%%%%%

\appendix[A] 
\appendixtitle{Calculation of Moist Adiabatic Profiles}
\label{app:calculation}
The moist adiabatic profiles are calculated numerically by assuming that saturation moist static energy $h$ is conserved, where:
\begin{equation}
h=c_p T+gz+L_v q^* \label{eq:mse}
\end{equation}
Here, $T$ is temperature, $z$ is height, $q^*$ is the saturation specific humidity, $g$ is the acceleration due to gravity, $c_p$ is the specific heat of dry air at constant pressure, and $L_v$ is the latent heat of vaporization. All thermodynamic constants are defined in Table~\ref{tab:tableA1}. Saturation vapor pressure is calculated using Eq.~\ref{eq:bolton} \citep{bolton1980}.

The calculation proceeds in discrete vertical steps of $\Delta p = 50$ Pa). For a given surface temperature ($T_s$) and surface pressure ($p_s$), $h$ is first calculated at the surface ($z=0$) and is held constant over height. At each subsequent pressure step $p_{i+1}$, the height $z_{i+1}$ is calculated using hydrostatic balance. Then, a numerical root-finding algorithm (scipy.optimize.root\_scalar with the Brentq method) is used to find the temperature $T_{i+1}$ that satisfies the condition that the $h$ at ($T_{i+1}, p_{i+1}, z_{i+1}$) is equal to the surface $h$.

To demonstrate that the non-monotonic warming is independent of the vertical coordinate, the results are also presented in height coordinates (Fig.~\ref{fig:fig-a1}). These profiles are obtained by following the same calculation as above except stepping in uniform intervals $\Delta z=100$ m. The pressure $p_{i+1}$ at height $z_{i+1}$ is calculated using hydrostatic balance.

\begin{figure}[htbp]
 \centering
 \includegraphics[width=\textwidth]{fig-a1.png}
 \caption{The moist adiabatic warming response to a 4 K surface warming in pressure coordinates. (a) Vertical profiles of the temperature response ($\Delta T$) as a function of pressure for surface temperatures ($T_s$) 280, 290, 300, 310, and 320 K. (b) The warming ($\Delta T$) at 5~km, 10~km, 15~km, and 20~km as a function of $T_s$. The non-monotonic behavior seen in height coordinates (Fig.~\ref{fig:fig-1}c) is also evident in pressure coordinates.}\label{fig:fig-a1}
\end{figure}


\begin{table}[htbp]
\caption{Thermodynamic constants used in the calculation of moist adiabatic profiles.}\label{tab:tableA1}
\begin{center}
\begin{tabular}{llcl}
\topline
Symbol & Description & Value & Units\\
\midline
$g$ & Acceleration due to gravity & 9.81 & m s$^{-2}$ \\
$c_p$ & Specific heat of dry air & 1005.7 & J kg$^{-1}$ K$^{-1}$ \\
$R_d$ & Gas constant for dry air & 287.05 & J kg$^{-1}$ K$^{-1}$ \\
$R_v$ & Gas constant for water vapor & 461.5 & J kg$^{-1}$ K$^{-1}$ \\
$\epsilon$ & Ratio of gas constants ($R_d/R_v$) & 0.622 & dimensionless \\
$p_s$ & Surface pressure & 1000 & hPa \\
$L_v$ & Latent heat of vaporization & $2.501 \times 10^6$ & J kg$^{-1}$ \\
\botline
\end{tabular}
\end{center}
\end{table}

\appendix[B] 
\appendixtitle{Effect of Latent Heat of Fusion on Moist Adiabatic Warming}
\label{app:fusion}
We assess how latent heat of fusion influences the non-monotonicity of moist adiabatic warming. We follow the IFS Cycle~40 approximations as summarized by \cite{flannaghan2014}. The fraction of liquid water $\alpha$ varies with $T$ as follows:
\begin{equation}
\alpha(T)=
\begin{cases}
0, & T \le T_{\mathrm{ice}},\\
\left(\dfrac{T-T_{\mathrm{ice}}}{T_0-T_{\mathrm{ice}}}\right)^2 & T_{\mathrm{ice}}<T<T_0,\\
1 & T \ge T_0,
\end{cases}
\end{equation}
where $T_{\mathrm{ice}}=253.15$ K and $T_0=273.15$ K. Thus all condensate is ice below 253.15 K, all condensate is liquid above 273.15 K, and a quadratic transition occurs in between.

The saturation vapor pressure $e^*$ is the weighted average over liquid ($e_\ell^*$) and ice ($e_i^*$):
\begin{equation}
e^*=\alpha e_{\ell}^*+(1-\alpha)e_i^*
\end{equation}
The saturation vapor pressure over liquid and ice is:
\begin{equation}
e_{\ell,i}^*(T) = a_1 \exp \left( a_3 \,\frac{T - T_0}{\,T - a_4\,} \right)
\label{eq:es_general}
\end{equation}
where over liquid $a_1=611.21$ Pa, $a_3=17.502$, $a_4=32.19$ K \citep{buck1981} and over ice $a_1=611.21$ Pa, $a_3=22.587$, $a_4=-0.7$ K \citep{alduchov1996}.

The effective latent heat of vaporization $L_e^*(T)$ includes both condensation and fusion:
\begin{equation}
L_e^*(T) = L_v + (1-\alpha) L_f
\end{equation}
where $L_f = 0.334 \times 10^6$~J~kg$^{-1}$ is the latent heat of fusion.

Moist adiabats including fusion are obtained by solving for $T$ that conserves moist static energy with the effective latent heat $L_e$:
\begin{equation}
    h = c_{pd}T + gz + L_e q^*  \label{eq:mse_fusion}
\end{equation}

The vertical profiles of warming $\Delta T$ and the warming at fixed pressure levels versus surface temperature exhibit similar non-monotonic behavior to the case without fusion (compare Fig.~\ref{fig:fig-1} and \ref{fig:fig-b1}). Latent heat of fusion introduces a secondary local maximum in the warming in the mid troposphere (500~hPa) due to the additional energy release from fusion. When the secondary peak is to the right of the primary peak the $T_s$ corresponding to peak warming shifts to colder $T_s$ with fusion (points below the 1:1 line in Fig.~\ref{fig:fig-b1}). As the secondary peak overlaps with the primary peak the $T_s$ corresponding to peak warming shifts to warmer $T_s$ with fusion (points above the 1:1 line in Fig.~\ref{fig:fig-b1}). This effect is greatest (6.03~K) at 727~hPa. Since fusion represents a secondary effect and complicates analytical treatment, we neglect it for the rest of the paper.

\begin{figure}[htbp]
 \centering
 \includegraphics[width=0.55\textwidth]{fig-b1.png}
 \caption{The moist adiabatic warming response to a 4 K surface warming with latent heat of fusion. (a) Vertical profiles of the temperature response ($\Delta T$) as a function of pressure for surface temperatures ($T_s$) of 280, 290, 300, 310, and 320 K. (b) The warming ($\Delta T$) at fixed pressure levels of 500, 400, 300, and 200 hPa as a function of $T_s$. (c) $T_s$ corresponding to peak warming with and without fusion.}\label{fig:fig-b1}
\end{figure}

\appendix[C] 
\appendixtitle{Effect of Saturation Vapor Pressure Formula on Moist Adiabatic Warming}
\label{app:svp}
The calculation of moist adiabatic warming profiles depends on the choice of the saturation vapor pressure formula. To assess the sensitivity of surface temperatures associated with peak moist adiabatic warming to different formula we test three formula: \cite{bolton1980}, Goff-Gratch \citep{list1949}, and \cite{murphy2005}.

The \cite{bolton1980} formula is:
\begin{equation}
e^* = 6.112 \exp\left(\frac{17.67 (T - 273.15)}{T - 29.65}\right) \quad [\text{hPa}],
\label{eq:bolton}
\end{equation}

The Goff-Gratch formula is:
\begin{multline}
\log_{10} e^* = -7.90298 \left(\frac{373.16}{T} - 1\right) + 5.02808 \log_{10}\left(\frac{373.16}{T}\right)\\ - 1.3816 \times 10^{-7} \left(10^{11.344 (1 - T/373.16)} - 1\right)\\ + 8.1328 \times 10^{-3} \left(10^{-3.49149 (373.16/T - 1)} - 1\right)\\ + \log_{10}(1013.246) \quad [\text{hPa}]
\end{multline}

The \cite{murphy2005} formula is:
\begin{multline}
\ln e^* = 54.842763 - \frac{6763.22}{T} - 4.210 \ln T + 0.000367 T\\ + \tanh\left(0.0415 (T - 218.8)\right) \left(53.878 - \frac{1331.22}{T} - 9.44523 \ln T + 0.014025 T\right) \quad [\text{Pa}],
\end{multline}
where $T$ is in Kelvin for all 3 formula.

\cite{bolton1980} is sufficiently accurate for the purposes of evaluating the $T_s$ that leads to maxima in moist adiabatic warming (Fig.~\ref{fig:fig-c1}). The differences in peak $T_s$ across the three saturation vapor pressure formula are small, with the largest deviation being 0.27~K between Bolton and Goff-Gratch and 0.34~K between Bolton and Murphy-Koop. Thus we use \cite{bolton1980} for the rest of the paper.

\begin{figure}[htbp]
 \centering
 \includegraphics[width=0.4\textwidth]{fig-c1.png}
 \caption{(a) $T_s$ corresponding to peak warming using \cite{bolton1980} and Goff-Gratch saturation vapor pressure formula. (b) Same as (a) but comparing \cite{bolton1980} and \cite{murphy2005}.}\label{fig:fig-c1}
\end{figure}

\clearpage

%%%%%%%%%%%%%%%%%%%%%%%%%%%%%%%%%%%%%%%%%%%%%%%%%%%%%%%%%%%%%%%%%%%%%
% REFERENCES
%%%%%%%%%%%%%%%%%%%%%%%%%%%%%%%%%%%%%%%%%%%%%%%%%%%%%%%%%%%%%%%%%%%%%
% This shows how to enter the commands for making a bibliography using
% BibTeX. It uses references.bib and the ametsocV6.bst file for the style.

\bibliographystyle{ametsocV6}
\bibliography{references}


\end{document}
%%%%%%%%%%%%%%%%%%%%%%%%%%%%%%%%%%%%%%%%%%%%%%%%%%%%%%%%%%%%%%%%%%%%%
% END OF AMSPAPERV6.1.TEX
%%%%%%%%%%%%%%%%%%%%%%%%%%%%%%%%%%%%%%%%%%%%%%%%%%%%%%%%%%%%%%%%%%%%%