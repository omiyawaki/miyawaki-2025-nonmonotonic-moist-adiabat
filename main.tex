%% Version 7.1, 1 September 2021
%
%%%%%%%%%%%%%%%%%%%%%%%%%%%%%%%%%%%%%%%%%%%%%%%%%%%%%%%%%%%%%%%%%%%%%%
% amspaperV6.tex --  LaTeX-based instructional template paper for submissions to the 
% American Meteorological Society
%
%%%%%%%%%%%%%%%%%%%%%%%%%%%%%%%%%%%%%%%%%%%%%%%%%%%%%%%%%%%%%%%%%%%%%
% PREAMBLE
%%%%%%%%%%%%%%%%%%%%%%%%%%%%%%%%%%%%%%%%%%%%%%%%%%%%%%%%%%%%%%%%%%%%%

%% Start with one of the following:
% 1.5-SPACED VERSION FOR SUBMISSION TO THE AMS
\documentclass[draft]{ametsocV6.1}

% TWO-COLUMN JOURNAL PAGE LAYOUT---FOR AUTHOR USE ONLY
% \documentclass[twocol]{ametsocV6.1}

%%%%%%%%%%%%%%%%%%%%%%%%%%%%%%%%

%%% To be entered by author:

%% May use \\ to break lines in title:

\title{The non-monotonicity of moist adiabatic warming}

%% Enter authors' names and affiliations as you see in the examples below.
%
%% Use \correspondingauthor{} and \thanks{} (\thanks command to be used for affiliations footnotes, 
%% such as current affiliation, additional affiliation, deceased, co-first authors, etc.)
%% immediately following the appropriate author.
%
%% Note that the \correspondingauthor{} command is NECESSARY.
%% The \thanks{} commands are OPTIONAL.
%
%% Enter affiliations within the \affiliation{} field. Use \aff{#} to indicate the affiliation letter at both the
%% affiliation and at each author's name. Use \\ to insert line breaks to place each affiliation on its own line.

\authors{Osamu Miyawaki\aff{a}\correspondingauthor{Osamu Miyawaki, miyawako@union.edu}}

\affiliation{\aff{a}{Department of Geosciences, Union College, Schenectady New York, USA}}

%%%%%%%%%%%%%%%%%%%%%%%%%%%%%%%%%%%%%%%%%%%%%%%%%%%%%%%%%%%%%%%%%%%%%
% ABSTRACT
%
% Enter your abstract here
% Abstracts should not exceed 250 words in length!
%

\abstract{The moist adiabat is a useful first-order approximation of the tropical stratification and thus governs fundamental properties of climate such as the static stability and the lapse rate feedback. While total atmospheric latent heating increases monotonically with warming, the resulting change in temperature along a moist adiabat is surprisingly non-monotonic with surface temperature. This phenomenon has lacked a physical explanation. This paper presents a thermodynamic explanation by decomposing the sensitivity of the moist adiabatic lapse rate into two competing components: 1) A Cooling Term arising from the partial derivative of saturation specific humidity with respect to temperature ($\partial q_s/\partial T$), which is proportional to $q_s/T^2$ via the Clausius-Clapeyron relation, and 2) a Pressure Term arising from the partial derivative with respect to pressure ($\partial q_s/\partial p$), which is proportional to $q_s/p$. The non-monotonicity arises because while both terms grow with temperature due to the exponential increase of saturation specific humidity ($q_s$), the $1/T^2$ prefactor on the Cooling Term suppresses its growth more strongly than the pressure-related prefactor on the Pressure Term. This mechanism also explains the non-monotonic behavior of convective buoyancy and vertical velocity.}

\begin{document}


%% Necessary!
\maketitle

%%%%%%%%%%%%%%%%%%%%%%%%%%%%%%%%%%%%%%%%%%%%%%%%%%%%%%%%%%%%%%%%%%%%%
% MAIN BODY OF PAPER
%%%%%%%%%%%%%%%%%%%%%%%%%%%%%%%%%%%%%%%%%%%%%%%%%%%%%%%%%%%%%%%%%%%%%
%
\section{Introduction}

The Clausius-Clapeyron relation describes the potential for a warmer atmosphere to hold more water vapor \citep{emanuel1994}. This principle is the basis for the positive water vapor feedback \citep{held2000a} and various scaling theories in response to warming including extreme precipitation \citep{oGorman2015} and CAPE \citep{romps2016}. 

In the tropics, convection couples the surface with the free troposphere. Although processes like convective entrainment influence the details of this coupling \citep{miyawaki2020}, moist adiabatic adjustment serves as a useful first-order approximation \citep{held1993}. The top-heavy warming profile predicted by moist adiabatic adjustment (Fig.~\ref{fig:fig-1}b) is a robust feature in climate models and observations, despite historical challenges in observational records \citep{vallis2015, santer2005}.

This warming profile is important because it increases atmospheric static stability, which influences convection \citep{neelin1987}. This structure also defines the tropical lapse rate feedback, a key negative feedback for global climate sensitivity \citep{hansen1984}. The lapse rate feedback partially cancels the water vapor feedback and scales in tandem because amplified warming in the upper troposphere is a consequence of enhanced latent heat release \citep{held2012}. In a moist adiabatic atmosphere that is saturated at the surface, total latent heat release scales with surface humidity for deep convection that reaches the tropopause where $q_\text{tropopause}\ll q_\text{surface}$ because the tropopause temperature is approximately invariant with warming \citep{seeley2019}. Given the monotonic increase in surface humidity (Fig.~\ref{fig:fig-1}a), one might expect moist adiabatic warming to also be monotonic with surface temperature at all heights. However, it is a non-monotonic function of surface temperature at a fixed pressure (Fig.~\ref{fig:fig-1}c, see Appendix~A for details on how the moist adiabat is calculated). This non-monotonicity arises in height coordinates (Fig.~\ref{fig:fig-a1}) and when the latent heat of fusion is included (see Appendix~B and Fig.~\ref{fig:fig-b1}). While \citet{levine2016} showed this non-monotonicity and its influence on zonal stationary circulations, a physical explanation for the non-monotonicity in moist adiabatic warming currently does not exist in the literature.

This raises the question: What physical mechanism drives this non-monotonic warming? This paper presents a thermodynamic explanation for the origins of non-monotonicity in moist adiabatic warming and its cascading effects on buoyancy and vertical velocity. Section 2 develops the theory of non-monotonic warming. Section 3 explores the implications of this non-monotonicity for buoyancy and vertical velocity. Section 4 provides a summary and discussion.

\begin{figure}[htbp]
 \centering
 \includegraphics[width=0.4\textwidth]{fig-1.png}\\
\caption{(a) The change in column-integrated saturation specific humidity resulting from a 4 K surface warming as a function of surface temperature ($T_s$). (b) Profiles of the moist adiabatic temperature response ($\Delta T$) to a 4 K surface warming, plotted against pressure, for $T_s = $ 280, 290, 300, 310, and 320 K. (c) The warming ($\Delta T$) at 500, 400, 300, and 200 hPa as a function of $T_s$, showing a non-monotonic response where warming peaks at an intermediate temperature.}
\label{fig:fig-1}
\end{figure}

\section{Theory of Non-Monotonic Warming}
We start by defining the moist adiabatic temperature profile in pressure coordinates $T(p)$ in terms of the moist adiabatic lapse rate $\Gamma_m = dT/dp$:
\begin{equation}
T(p) = T_s + \int_{p_s}^{p} \Gamma_m \, dp' \label{eq:temp_profile_pressure}
\end{equation}
where $T_s$ is surface temperature. The difference between a perturbed and baseline state ($\Delta$) then follows as
\begin{equation}
\Delta T(p) = \Delta T_s + \int_{p_s}^{p} \Delta\Gamma_m \, dp' \label{eq:delta_t_pressure}
\end{equation}
For a small perturbation, $\Delta \Gamma_m$ can be approximated using a first-order Taylor expansion: $\Delta\Gamma_m \approx \frac{d\Gamma_m}{dT_s}\Delta T_s$. Substituting this into Eq.~(\ref{eq:delta_t_pressure}) gives:
\begin{equation}
\Delta T(p) \approx \Delta T_s + \left(\int_{p_s}^{p} \frac{d\Gamma_m}{dT_s}dp'\right)\Delta T_s \label{eq:delta_t_taylor_pressure}
\end{equation}
Thus the non-monotonicity in moist adiabatic warming is encoded into $d\Gamma_m/dT_s$, the sensitivity of the moist adiabatic lapse rate to surface temperature. Indeed, $d\Gamma_m/dT_s$ is non-monotonic with respect to temperature (dashed line shows the local minima of $d\Gamma_m/dT_s$ in Fig.~\ref{fig:fig-2}a). Note that $d\Gamma_m/dT_s$ is mostly negative in the troposphere (Fig.~\ref{fig:fig-2}b). This is consistent with amplified warming aloft because the integral in Eq.~(\ref{eq:delta_t_pressure} is from high to low pressure, which introduces a negative sign.

$\Gamma_m$ is a function of local state variables $\Gamma_m(T, p)$. Thus to make progress in understanding $d\Gamma_m/dT_s$, we must rewrite $d\Gamma_m/dT_s$ in terms of derivatives with respect to the local state variables $(T, p)$. To do this we first use the chain rule: 
\begin{equation}
\frac{d\Gamma_m}{dT_s} = \left(\frac{\partial\Gamma_m}{\partial T}\right)_p \cdot \frac{dT}{dT_s} + \left(\frac{\partial\Gamma_m}{\partial p}\right)_T \cdot \frac{dp}{dT_s} \label{eq:chain_rule_start}
\end{equation}
The second term $\frac{dp}{dT_s}=0$ because pressure is the vertical coordinate and is an independent variable. Recognizing that by definition $\Gamma_m = \frac{dT}{dp}$,
\begin{equation}
    \frac{d}{dp}\left(\frac{dT}{dT_s}\right) = \left(\frac{\partial\Gamma_m}{\partial T}\right)_p \cdot \frac{dT}{dT_s} 
    \label{eq:ode}
\end{equation}
This is an ordinary differential equation for $\frac{dT}{dT_s}$ as a function of pressure. The solution with the boundary condition $\frac{dT}{dT_s}(p_s) = 1$, is:
\begin{equation}
    \frac{dT}{dT_s} = \exp\left(\int_{p_s}^{p} \left(\frac{\partial\Gamma_m}{\partial T}\right)_p dp'\right)
    \label{eq:ode-solution}
\end{equation}
Substituting Eq.~(\ref{eq:ode-solution}) back into Eq.~(\ref{eq:chain_rule_start}) gives:
\begin{equation}
\frac{d\Gamma_m}{dT_s} = \left(\frac{\partial\Gamma_m}{\partial T}\right)_p \cdot \exp\left(\int_{p_s}^{p} \left(\frac{\partial\Gamma_m}{\partial T}\right)_{p'} dp'\right) \label{eq:total_sensitivity}
\end{equation}
where $(\partial\Gamma_m/\partial T)_p$ is the moist adiabatic lapse rate sensitivity to local temperature $T$ at pressure level $p$. The integral describes how a surface temperature perturbation influences $\Gamma_m$ through the sum of all $\Gamma_m$ changes that occur below $p$.

The non-monotonicity can arise from either 1) $(\partial\Gamma_m/\partial T)_p$ being non-monotonic and the integral acting to amplify it or 2) $(\partial\Gamma_m/\partial T)_p$ being monotonic but sign changes in $(\partial\Gamma_m/\partial T)_p$ leads to the integral being non-monotonic. Numerical solutions show that $(\partial\Gamma_m/\partial T)_p$ is non-monotonic (dash-dot line shows the local minima of $d\Gamma_m/dT$ in Fig.~\ref{fig:fig-2}c), which is further amplified by the integral term (Fig.~\ref{fig:fig-2}d).

Why is $(\partial\Gamma_m/\partial T)_p$ non-monotonic with $T$? To understand this we solve for $\Gamma_m$ from the first law of thermodynamics for adiabatic ascent with latent heating assuming the parcel is saturated:
\begin{equation}
c_{p} dT - \alpha dp + L_v dq_s = 0 \label{eq:mse_pressure}
\end{equation}
where $c_{p}$ is the specific heat capacity of air at constant pressure, $\alpha$ is specific volume, $L_v$ is the latent heat of vaporization, and $q_s$ is the saturation specific humidity. We assume 1) $c_p \approx c_{pd}$, neglecting the role of water of all phases on the specific heat capacity and 2) $\alpha \approx \alpha_d = R_d T/p$, neglecting the virtual effect of vapor on density. 

Use the chain rule to expand $dq_s$:
\begin{equation}
dq_s = \left(\frac{\partial q_s}{\partial T}\right)_p dT + \left(\frac{\partial q_s}{\partial p}\right)_T dp \label{eq:dqs_expansion}
\end{equation}

Substituting Eq.~(\ref{eq:dqs_expansion}) into Eq.~(\ref{eq:mse_pressure}) and rearranging gives
\begin{equation}
\left(c_{pd} + L_v\left(\frac{\partial q_s}{\partial T}\right)_p \right)dT = \left(\alpha_d - L_v\left(\frac{\partial q_s}{\partial p}\right)_T\right)dp \label{eq:rearranged}
\end{equation}
We can derive closed-form expressions for the $q_s$ derivatives using the Clausius-Clapeyron relation and Dalton's Law. These $q_s$ derivatives describe the role of phase equilibrium shifts in $q_s$ with $T$ and $p$ on the effective heat capacity and specific volume of the air parcel, respectively:
\begin{align}
c_L &\equiv L_v\left(\frac{\partial q_s}{\partial T}\right)_p \approx \frac{L_v^2 q_s}{R_v T^2}
\label{eq:c_L} \\
\alpha_L &\equiv -L_v\left(\frac{\partial q_s}{\partial p}\right)_T \approx \frac{L_v q_s}{p}
\label{eq:alpha_L}
\end{align}
where the approximation arises from assuming saturation vapor pressure $e_s \ll p$.

$c_L$ can be thought of as a latent heat capacity, representing the enhanced thermal inertia due to the fact that latent heating buffers some of the cooling from expansion. Thus $c_L$ acts to increase the heat capacity of the air parcel such that it has an effective heat capacity $c_{pd} + c_L$.

$\alpha_L$ can be thought of as a latent specific volume, representing the enhanced expansion of air with ascent due to the fact that lower pressure shifts the phase equilibrium of water to favor the vapor phase over liquid. Thus $\alpha_L$ acts to increase the volume of air such that it has an effective specific volume $\alpha_d + \alpha_L$.

Now solving for the moist adiabatic lapse rate $\Gamma_m = dT/dp$:
\begin{align}
\Gamma_m = \frac{dT}{dp} &= \frac{\alpha_d +\alpha_L}{c_{pd} + c_L} \label{eq:gamma_m_ratio} \\
&= \Gamma_d \cdot \frac{1+\frac{\alpha_L}{\alpha_d}}{1+\frac{c_L}{c_{pd}}} \label{eq:gamma_m_factored}
\end{align}
where $\Gamma_d = \alpha_d / c_{pd}$ is the dry adiabatic lapse rate in pressure coordinates and the two non-dimensional terms represent the fractional increase in effective heat capacity and specific volume due to phase equilibrium changes:
\begin{align}
\tilde{c} &= \frac{c_L}{c_{pd}} = \frac{L_v^2 q_s}{c_{pd} R_v T^2} \label{eq:c_ratio} \\
\tilde{\alpha} &= \frac{\alpha_L}{\alpha_d} = \frac{L_v q_s}{R_d T} = \frac{R_v c_{pd}T}{R_dL_v}\tilde{c} = k\tilde{c} \label{eq:alpha_ratio}
\end{align}
Substituting Eq.~(\ref{eq:c_ratio}) and Eq.~(\ref{eq:alpha_ratio}) into Eq.~(\ref{eq:gamma_m_factored}) gives:
\begin{equation}
\Gamma_m = \Gamma_d \cdot \frac{1 + k\tilde{c}}{1 + \tilde{c}} \label{eq:gamma_m_tilde}
\end{equation}

For typical values in Earth's atmosphere ($R_v=461$ J kg$^{-1}$ K$^{-1}$, $R_d=287$ J kg$^{-1}$ K$^{-1}$, $c_{pd}=1005$ J kg$^{-1}$ K$^{-1}$, $L_v=2.5\times10^6$ J kg$^{-1}$, and $T \in [200, 320]$ K), the factor $k=\frac{R_v c_{pd}T}{R_dL_v}\in [0.13, 0.21]$. Thus $k$ is a weak function of temperature and is a quasi-constant of order $10^{-1}$. In contrast, $\tilde{c}$ scales exponentially with temperature (through $q_s$) and varies from $\tilde{c}(200\text{ K})\sim 10^{-4}$ to $\tilde{c}(320\text{ K})\sim 10^{1}$. Thus the temperature sensitivity of $\Gamma_m$ is controlled by $\tilde{c}$. Because $\Gamma_m$ is bounded between $\Gamma_d$ (dry limit, $\tilde{c} \to 0$) and $k\Gamma_d$ (moist limit, $\tilde{c} \to \infty$), the magnitude of $\partial\Gamma_m/\partial T$ must peak at some intermediate $\tilde{c}$ else $\Gamma_m$ would be unbounded.

Where does the magnitude of $\partial\Gamma_m/\partial T$ reach its peak value? To solve this we use the quotient rule on Eq.~(\ref{eq:gamma_m_ratio}):
\begin{equation}
\frac{\partial\Gamma_m}{\partial T} = \underbrace{\frac{1}{c_{pd} + c_L}\frac{\partial(\alpha_d + \alpha_L)}{\partial T}}_{\text{latent volume sensitivity}} + \underbrace{\left(-\frac{\alpha_d + \alpha_L}{(c_{pd} + c_L)^2}\frac{\partial c_L}{\partial T}\right)}_{\text{latent heat capacity sensitivity}} \label{eq:decomposition}
\end{equation}
The latent volume sensitivity varies monotonically with $T_s$ (Fig.~\ref{fig:fig-3}a, c). The latent heat capacity sensitivity varies non-monotonically with $T_s$ (Fig.~\ref{fig:fig-3}b, d). Thus we further decompose the latent heat capacity sensitivity to probe its origin:
\begin{equation}
-\frac{\alpha_d + \alpha_L}{(c_{pd} + c_L)^2}\frac{\partial c_L}{\partial T} = -\frac{1}{p} \cdot \left(1 + \tilde{\alpha}\right) \cdot \frac{R_d}{c_{pd}}\frac{\partial\log{c_L}}{\partial \log{T}} \cdot f_d \cdot f_L \label{eq:term_b_intermediate}
\end{equation}
where
\begin{equation}
f_d \equiv c_{d}/(c_{pd} + c_L) \label{eq:f_d}
\end{equation}
\begin{equation}
f_L \equiv c_{L}/(c_{pd} + c_L) \label{eq:f_L}
\end{equation}
and $f_d + f_L = 1$. $f_d$ and $f_L$ represent the dry and latent fractions of effective heat capacity. 

Eq.~(\ref{eq:term_b_intermediate}) shows the latent heat capacity sensitivity is a product of four terms that vary monotonically with $T$. $\tilde{\alpha}=L_v q_s / (\alpha_d p)$ scales exponentially with $T$ through $q_s$ (red line in Fig.~\ref{fig:fig-4}a). The fractional change in latent heat capacity to a fractional change in temperature $\partial\log{c_L}/\partial\log{T} = L_v / (R_v T) - 2$ so it weakly decreases with $T$ (blue line in Fig.~\ref{fig:fig-4}a). The product of these two terms is weakly non-monotonic in $T$ with a local minimum where $\tilde{\alpha} \approx R_vT/L_v$ (white line in Fig.~\ref{fig:fig-4}b). At low $T$, $\tilde{\alpha}$ is small so the product is dominated by the decrease in $\partial\log{c_L}/\partial\log{T}$. At high $T$, $\tilde{\alpha}$ is large so the product is dominated by the exponential increase in $\tilde{\alpha}$ . However, the non-monotonicity of these two terms are not the source of the peak in the magnitude of $\partial\Gamma_m/\partial T$, which requires a local maximum, not a minimum.

The dry fraction of effective heat capacity $f_d=c_{pd}/(c_{pd}+c_L)$ logistically decreases with $T$ because $c_{pd}$ is a constant while latent heat capacity $c_L$ increases exponentially with $T$ through $q_s$ (blue line in Fig.~\ref{fig:fig-4}c). The latent fraction of effective heat capacity $f_L = c_L / (c_{pd}+c_L)$ logistically increases with $T$ (red line in Fig.~\ref{fig:fig-4}c). The product $f_d\cdot f_L$ is maximized when $f_d=f_L$, or $c_L = c_{pd}$ (black line in Fig.~\ref{fig:fig-4}d).

What is the physical intuition behind the peak at $c_L = c_{pd}$? Recall that $c_L$ quantifies the enhancement of effective heat capacity due to latent heat of condensation offsetting adiabatic cooling. The $q_s$ derivative in $c_L$ requires two ingredients: 1) cooling from expansion and 2) water vapor. $f_d$ and $f_L$ represent the fractional availability of the two ingredients. At low $T$, condensation is limited by the availability of water vapor (red line in Fig.~\ref{fig:fig-4}c). At high $T$ condensation is limited by adiabatic cooling (blue line in Fig.~\ref{fig:fig-4}c). The peak in latent heat capacity sensitivity corresponds to where the availability of cooling and vapor are equally limiting (black line in Fig.~\ref{fig:fig-4}c). Thus the non-monotonicity in $\partial\Gamma_m/\partial T$ and moist adiabatic warming arises from the competition between the two limiting factors of condensation.

How well does the condition $c_L = c_{pd}$ capture the actual peak in $\partial\Gamma_m/\partial T$? The theory slightly overpredicts the $T_s$ where the magnitude of $\partial\Gamma_m/\partial T$ peaks (compare solid and dash-dot lines in Fig.~\ref{fig:fig-5}). This error is due to the weak non-monotonicity in the product $(1+\tilde{\alpha})R_d/c_{pd}\partial\log(c_L)/\partial\log(T)$ which decreases with pressure (Fig.~\ref{fig:fig-4}b). The error maximizes at the surface where the theory predicts a peak $T_s$ that is 1.6 K warmer than the true peak $T_s$. 

The error in $T_s$ predicted by the theory and the true peak of $\Gamma_m / d T_s$ grows with height because the integral term in Eq.~(\ref{eq:total_sensitivity}) amplifies the error in $\partial\Gamma_m / \partial T$ at each level below. This error maximizes at 420 hPa where $c_L = c_{pd}$ predicts a peak $T_s$ that is 2.0 K warmer than the true peak $T_s$ (compare solid and dashed lines in Fig.~\ref{fig:fig-5}). This error is further compounded for $T_s$ corresponding to the peak of moist adiabatic warming $\Delta T$ (Eq.~\ref{eq:delta_t_taylor_pressure}), leading to a maximum error of 6.6 K at 420 hPa (compare solid and dotted lines in Fig.~\ref{fig:fig-5}). Thus the condition $c_L = c_{pd}$ provides a useful first-order estimate of $T_s$ where moist adiabatic warming peaks. Importantly the theory successfully captures the shift to warmer peak $T_s$ with height, which is due to the fact that temperature decreases with height and thus the transition from the vapor limited to cooling limited regime occurs at a warmer surface temperature with height.

\begin{figure}[htbp]
 \centering
 \includegraphics[width=\textwidth]{fig-2.png}\\
 \caption{The sensitivity of the moist adiabatic lapse rate to a change in surface temperature, $\partial\Gamma_m/\partial T_s$, exhibits a non-monotonic structure as a function of height and temperature (a). This structure arises from the competition between two opposing physical effects. The Cooling Term (b), which represents the effect of condensation from adiabatic cooling, is a negative contribution at all temperatures. The Pressure Term (c), which represents the effect of decreasing pressure with height, is a positive contribution. The non-monotonicity of the total sensitivity arises because the positive contribution from the Pressure Term grows more rapidly with $T_s$ than the negative contribution from the Cooling Term.}\label{fig:fig-2}
\end{figure}

\begin{figure}[htbp]
 \centering
 \includegraphics[width=\textwidth]{fig-3.png}\\
 \caption{Warming is decomposed into contributions from the Cooling Term and the Pressure Term. (a) The vertical profile of the warming contribution from the Cooling Term for select $T_s$. (b) The warming contribution from the Cooling Term at fixed heights as a function of surface temperature. This term provides a warming effect that increases monotonically with temperature. (c) The vertical profile of the relative cooling contribution from the Pressure Term. (d) The relative cooling from the Pressure Term at fixed heights. Both the Cooling and Pressure terms become stronger as the surface temperature increases.}\label{fig:fig-3}
\end{figure}

\begin{figure}[htbp]
 \centering
 \includegraphics[width=\textwidth]{fig-4.png}\\
 \caption{The (a) Cooling Prefactor, $\Gamma_m L_v / (R_v T^2)$, and (b) Pressure Prefactor, $g/(R_d T_v)$, as a function of height and surface temperature. The Cooling Prefactor weakens strongly with temperature due to its $1/T^2$ dependence. In contrast, the Pressure Prefactor weakens more slowly due to its $1/T_v$ dependence.}\label{fig:fig-4}
\end{figure}


\section{Implications of non-monotonicity in moist adiabatic warming on convection}
The non-monotonic warming of a moist adiabat has implications for the dynamics of convection. For example, \cite{romps2016} showed that parcel buoyancy is a non-monotonic function of surface temperature. Specifically the criterion where $B$ peaks is $\beta = 2c_{pd}$ where
\begin{equation}
\beta = c_{pd} + L_v\frac{\partial q_s}{\partial T} = c_{pd} + c_L
\end{equation}
Thus the criterion that maximizes $B$ is equivalent to where moist adiabatic warming peaks, $c_{pd} = c_L$. Below, we show this is true if the entrainment parameter $a = PE \epsilon / g$\footnote{$PE$ is precipitation efficiency, $\epsilon$ is the fractional entrainment rate, and $g$ is gravitational acceleration. See \cite{romps2016} for the derivation of the entraining plume model.} is small and derive a more general criterion that maximizes $B$. 

Buoyancy $B$ is the normalized virtual temperature (or equivalently, density) difference between the rising parcel $T_{v,p}$ and the environment $T_{v,e}$. Here we neglect the virtual effects of water and we use standard temperature:
\begin{equation}
B\approx\frac{g}{T_e}(T_p-T_e) \label{eq:buoyancy_def}
\end{equation}
As before, we express temperature profiles in terms of $T_s$ and the integral of their respective lapse rates. We assume the parcel follows a moist adiabatic lapse rate, $\Gamma_m$, while the environment follows an entraining lapse rate, $\Gamma_e$.
\begin{align}
T_p&=T_s+\int_{p_s}^p \Gamma_m(p') \, dp' \label{eq:Tp_profile} \\
T_e&=T_s+\int_{p_s}^p \Gamma_e(p') \, dp' \label{eq:Te_profile}
\end{align}
Substituting Eq.~(\ref{eq:Tp_profile}) and (\ref{eq:Te_profile}) into the definition of buoyancy Eq.~(\ref{eq:buoyancy_def}) yields:
\begin{equation}
B\approx\frac{g}{T_e}\int_{p_s}^p \delta \Gamma \, dp' \label{eq:buoyancy_integral}
\end{equation}
where $\delta\Gamma = \Gamma_e - \Gamma_m$. We use the same entraining plume model as in \cite{romps2016} but express the lapse rate in pressure coordinates:
\begin{equation}
\Gamma_e = \Gamma_d \cdot \frac{(1+a)\alpha_d + \alpha_L}{(1+a)c_{pd}+c_L} \label{eq:gamma_e}
\end{equation}
Substituting Eq.~(\ref{eq:gamma_m_ratio}) and (\ref{eq:gamma_e}) into Eq.~(\ref{eq:buoyancy_integral}) and simplifying gives:
\begin{equation}
    B = \frac{g}{T_e}\int_{p_s}^p \Gamma_d \cdot \frac{a(1-k)\tilde{c}}{(1+a+\tilde{c})(1+\tilde{c})} \, dp' \label{eq:buoyancy_final}
\end{equation}
If we assume that $a$ does not vary with $T_s$, $T_e$ increases monotonically with $T_s$ at all $p$. The origin of the non-monotonicity of $B$ must be in the integrand, $\delta \Gamma$. $B$ depends on $T$ primarily through $\tilde{c}$, which scales exponentially with $T$ through $q_s$, whereas $\Gamma_d$ and $k$ are linear functions of $T$. In the limit of $\tilde{c} \to 0$ (cold and dry), $\delta\Gamma$ scales as $\tilde{c}$, which increases with $T$. In the limit of $\tilde{c} \to \infty$ (warm and humid), $\delta\Gamma$ scales as $\tilde{c}^{-1}$, which decreases with increasing $T$. Thus the integrand maximizes at some intermediate $\tilde{c}$.

To solve for the condition that maximizes buoyancy we solve for the $\tilde{c}$ derivative of the integrand $\delta \Gamma$ in Eq.~(\ref{eq:buoyancy_final}) and set it to zero:
\begin{equation}
    \frac{d}{d \tilde{c}}\left(\Gamma_d \cdot \frac{a(1-k)\tilde{c}}{(1+a+\tilde{c})(1+\tilde{c})}\right) = 0 \label{eq:buoyancy_derivative}
\end{equation}
If we assume that $a$, $k$, and $\Gamma_d$ do not vary with $T$, the solution to Eq.~(\ref{eq:buoyancy_derivative}) is
\begin{equation}
    \tilde{c}_\text{peak}=\sqrt{1+a} \label{eq:buoyancy_quadratic}
\end{equation}
Thus the condition that maximizes buoyancy is $c_L = \sqrt{1+a} c_{pd}$. In the limit of weak entrainment $a \to 0$, this reduces to $c_L = c_{pd}$. In the presence of entrainment, buoyancy peaks at a higher $c_L$ and thus higher $T_s$ all else equal. Entrainment reduces the latent heat released by the cooling parcel given the same $q_s$ so it shifts the critical point that separates the vapor limited and cooling limited regimes toward higher $q_s$.

How important is the factor $\sqrt{1+a}$? For an entrainment rate representative of Earth's current climate $a=0.2$, the difference in peak $T_s$ that corresponds to $c_L=c_{pd}$ and $c_L=\sqrt{1+a}c_{pd}$ are $< 1.49$ K (compare red and solid black line in Fig.~\ref{fig:fig-6}a). This difference decreases with height and becomes negligibly small around the tropopause (e.g., 0.33 K at $p=100$ hPa), which explains why the criteria $c_L = c_{pd}$ works well for explaining the non-monotonicity of CAPE \citep{romps2016}. However, for stronger entrainment rates and for understanding the non-monotonicity of buoyancy in the lower troposphere the factor $\sqrt{1+a}$ can be important (e.g., 4.38 K for $a=0.7$ at the surface; compare red and solid black line in Fig.~\ref{fig:fig-6}b).


How well do these criteria capture the $T_s$ that maximizes buoyancy across the troposphere? We will first focus on $\delta \Gamma$, i.e. the integrand in Eq.~(\ref{eq:buoyancy_integral}). For $a=0.2$ both criteria capture the $T_s$ that corresponds to the peak in $\delta \Gamma$ well ($<1.39$ K for $c_L=\sqrt{1+a}c_{pd}$, $< 2.87$ K for $c_L=c_{pd}$, compare red and solid black line to dashed line in Fig.~\ref{fig:fig-6}a). The small error arises even for the $c_L=\sqrt{1+a}c_{pd}$ criterion because $\Gamma_d(1-k)$ is weakly non-monotonic with $T$ ($\Gamma_d$ increases with $T$ and $(1-k)$ decreases with $T$), which we ignored in order to analytically solve Eq.~(\ref{eq:buoyancy_derivative}). This error is amplified as we integrate $\delta \Gamma$ to obtain buoyancy Eq.~(\ref{eq:buoyancy_integral}) because the errors in the location of peak $\delta \Gamma$ from each level below accumulates for the location of peak $B$ compare red and solid black line to dotted line in Fig.~\ref{fig:fig-6}a).

For a higher entrainment parameter $a=0.7$ the importance of the factor $\sqrt{1+a}$ becomes clear. The error in $T_s$ that corresponds to the peak in $\delta \Gamma$ is $<3.39$ K for the $c_L=\sqrt{1+a}c_{pd}$ criterion compared to $<5.83$ K for the $c_L=c_{pd}$ criterion (compare red and solid black line to dashed line in Fig.~\ref{fig:fig-6}b). The error in $T_s$ that corresponds to the peak in buoyancy is surprisingly lower for the $c_L=c_{pd}$ criterion ($<3.37$ K) compared to the $c_L=\sqrt{1+a}c_{pd}$ criterion ($<4.66$ K, compare red and solid black lines to dotted black line in Fig.~\ref{fig:fig-6}b). This is because $c_L=c_{pd}$ underpredicts $T_s$ for peak $B$ in the lower troposphere, which offsets the growth of the larger error in peak $\delta \Gamma$ (compare solid black and dotted lines in Fig.~\ref{fig:fig-6}b). While the criteria $c_L=c_{pd}$ may provide a better estimate of peak buoyancy in some cases, it doesn't do so for the right reasons. For example the criteria $c_L=c_{pd}$ predicts no shift in $T_s$ that maximizes $B$ to perturbations in $a$ while the criterion $c_L=\sqrt{1+a}c_{pd}$ captures the shift in peak $\delta \Gamma$ and $B$ toward warmer $T_s$ with increasing entrainment (Fig.~\ref{fig:fig-6}c).

This non-monotonic behavior of buoyancy extends to the strength of the convective updraft. We model the updraft's specific kinetic energy, $\frac{1}{2}w^2$, using Eq.~(1) from \cite{delgenio2007}:
\begin{equation}
\frac{d}{dz}\left(\frac{1}{2}w^2\right)=a'B(z)-(1+b')\epsilon(z)w^2 \label{eq:momentum}
\end{equation}
where $a'$ and $b'$ are dimensionless constants. We use $a'=1/6$ and $b'=2/3$ following \cite{delgenio2007}. $\epsilon(z)$ is the fractional entrainment rate, which is calculated following Eq.~(3) in \cite{romps2016} with entrainment parameter $a=0.2$ and precipitation efficiency $PE=0.35$. Since $w(z)$ is determined by the integral of the net force, which includes buoyancy, we expect the non-monotonic dependence on $T_s$ extends to the vertical velocity profile as well.

Numerically integrating Eq.~(\ref{eq:momentum}) confirms this expectation. The resulting vertical velocity varies non-monotonically with $T_s$ (Fig.~\ref{fig:fig-7}b). This leads w(z) becoming more top-heavy with warming, i.e. w decreases in the lower troposphere and increases in the upper troposphere (Fig.~\ref{fig:fig-7}a).

Are these findings relevant to Earth's atmosphere, where convection is not strictly moist adiabatic and vertical velocity is subject to details and constraints not considered here such as cloud microphysics and radiative cooling? To test this we analyzed output from a set of 9 convective-resolving models simulating radiative convective equilibrium in a 100 km x 100 km domain from the RCEMIP project \citep{wing2018}. We look at the mean vertical velocity profiles for $w$ exceeding the 99.9th percentile at each height level. The 99.9th percentile corresponds to the fastest 1000 samples of $w$ per level per model. We focus on strong convective updrafts because the buoyancy is highest for those parcels that are closest to the moist adiabat.

The vertical velocity profiles from the RCEMIP simulations show diverse $w_{>99.9}$ responses to variations in surface temperature (295, 300, and 305 K, see Fig.~\ref{fig:fig-8}). Some models exhibit a clear top-heavy shift in $w_{>99.9}$ with warming (e.g., CM1, DAM, UCLA-CRM, UKMO, WRF) accompanied by a decrease in $w_{>99.9}$ in the lower troposphere that is qualitatively consistent with the moist adiabatic theory (Fig.~\ref{fig:fig-7}a). SAM shows a top-heavy shift in $w_{>99.9}$ without a clear decrease in $w_{>99.9}$ in the lower troposphere. In the remaining models the $w_{>99.9}$ response exhibits non-monotonicity with $T_s$ but the peak $w_{>99.9}$ does not necessarily increase. For example DALES and SCALE predict a non-monotonic response in $w_{>99.9}$ with $T_s$ at $z\approx8$ km but the peak $w_{>99.9}$ weakens from $T_s=300$ to 305 K. MesoNH also predicts a decrease in peak $w_{>99.9}$ from $T_s=300$ K to 305 K but predicts a non-monotonic response in $w_{>99.9}$ with $T_s$ at $z\approx3$ km, much lower than in DALES and SCALE. The diversity in responses likely arises from differences in model details and emergent behavior such as convective organization that influence convective dynamics beyond the thermodynamic processes considered here. Nonetheless, the presence of non-monotonicity and a top-heavy shift in several models suggest that the implications of non-monotonicity in moist adiabatic warming on convective dynamics may be playing a role in shaping the response of convective updrafts in the real atmosphere.

\begin{figure}[htbp]
 \centering
 \includegraphics[width=\textwidth]{fig-5.png}\\
 \caption{(a) Vertical profiles of buoyancy for an undiluted parcel ascending through an environment set by an entraining plume, calculated for several surface temperatures. (b) Buoyancy at fixed heights as a function of surface temperature. The entraining environmental profile follows \cite{romps2016}.}\label{fig:fig-5}
\end{figure}

\begin{figure}[htbp]
 \centering
 \includegraphics[width=\textwidth]{fig-6.png}\\
 \caption{The total buoyancy from Fig.~\ref{fig:fig-5} is decomposed into contributions from the Cooling and Pressure terms. (a,b) The contribution to buoyancy from the Cooling Term, which provides a positive, monotonically increasing forcing. (c,d) The contribution from the Pressure Term, which provides a negative (suppressing) forcing that grows in magnitude with temperature.}\label{fig:fig-6}
\end{figure}

\begin{figure}[htbp]
 \centering
 \includegraphics[width=\textwidth]{fig-7.png}\\
 \caption{(a) Vertical profiles of updraft velocity, calculated by numerically integrating Eq.~(\ref{eq:momentum}) using the total buoyancy from Fig.~\ref{fig:fig-5}. (b) Updraft velocity at fixed heights as a function of surface temperature. The velocity exhibits a clear non-monotonic dependence on surface temperature, consistent with the behavior of buoyancy.}\label{fig:fig-7}
\end{figure}

\clearpage

\begin{figure}[htbp]
 \centering
 \includegraphics[width=\textwidth]{fig-8.png}\\
 \caption{The total vertical velocity is decomposed to show the influence of the Cooling and Pressure terms. (a,b) The velocity profile resulting from the positive buoyancy of the Cooling Term alone. (c,d) The effect of the Pressure Term on velocity, calculated as the residual between the total velocity and the velocity from the Cooling Term.}\label{fig:fig-8}
\end{figure}


\section{Summary and Discussion}

This paper presents a thermodynamic explanation for the non-monotonicity of moist adiabatic warming. The non-monotonicity arises through 

Our findings on buoyancy complement the work of \cite{romps2016}, who first explained the non-monotonicity of CAPE. The two studies offer different but complementary insights. \cite{romps2016} focused on explaining the non-monotonicity of buoyancy at the tropopause as a proxy for CAPE. Here, we focus on explaining the non-monotonicity of buoyancy at any fixed height. We also provide a different perspective on the source of non-monotonicity that arises from the competition in the sensitivity of a Cooling Term that favors condensation and a Pressure Term, driven by decreasing ambient pressure, that opposes it.

The non-monotonicity of moist adiabatic warming may have additional implications for climate, such as the organization of convection and the large-scale circulation response to warming. The non-monotonicity of moist adiabatic warming would drive a non-monotonic change in the meridional and zonal temperature gradients. This could serve as a thermodynamically driven hypothesis for understanding state dependence in the response of Hadley and Walker Cells to warming.

%%%%%%%%%%%%%%%%%%%%%%%%%%%%%%%%%%%%%%%%%%%%%%%%%%%%%%%%%%%%%%%%%%%%%
% ACKNOWLEDGMENTS
%%%%%%%%%%%%%%%%%%%%%%%%%%%%%%%%%%%%%%%%%%%%%%%%%%%%%%%%%%%%%%%%%%%%%
\acknowledgments
I thank Andrew Williams, Jiawei Bao, Jonah Bloch-Johnson, Martin Singh, and Stephen Po-Chedley for helpful discussions.

%%%%%%%%%%%%%%%%%%%%%%%%%%%%%%%%%%%%%%%%%%%%%%%%%%%%%%%%%%%%%%%%%%%%%
% DATA AVAILABILITY STATEMENT
%%%%%%%%%%%%%%%%%%%%%%%%%%%%%%%%%%%%%%%%%%%%%%%%%%%%%%%%%%%%%%%%%%%%%
% 
%
\datastatement
All scripts used for analysis and plots in this paper are available at \url{https://github.com/omiyawaki/miyawaki-2025-nonmonotonic-moist-adiabat}. They will also be archived on Zenodo upon publication.


%%%%%%%%%%%%%%%%%%%%%%%%%%%%%%%%%%%%%%%%%%%%%%%%%%%%%%%%%%%%%%%%%%%%%
% APPENDIXES
%%%%%%%%%%%%%%%%%%%%%%%%%%%%%%%%%%%%%%%%%%%%%%%%%%%%%%%%%%%%%%%%%%%%%

\appendix[A] 
\appendixtitle{Calculation of Moist Adiabatic Profiles}

The moist adiabatic profiles are calculated numerically by assuming that moist static energy (MSE) is conserved, where:
\begin{equation}
\text{MSE}=c_p T+gz+L_v q_s \label{eq:mse_appendix}
\end{equation}
Here, $T$ is temperature, $z$ is height, $q_s$ is the saturation specific humidity, $g$ is the acceleration due to gravity, $c_p$ is the specific heat of dry air at constant pressure, and $L_v$ is the latent heat of vaporization. All thermodynamic constants are defined in Table~\ref{tab:tableA1}. Saturation vapor pressure is calculated using Eq.~(10) in \cite{bolton1980}.

The calculation proceeds in discrete vertical steps of $\Delta p = 50$ Pa). For a given surface temperature ($T_s$) and surface pressure ($p_s$), MSE is first calculated at the surface ($z=0$) and is held constant over height. At each subsequent pressure step $p_{i+1}$, the height $z_{i+1}$ is calculated using hydrostatic balance. Then, a numerical root-finding algorithm (scipy.optimize.root\_scalar with the Brentq method) is used to find the temperature $T_{i+1}$ that satisfies the condition that the MSE at ($T_{i+1}, p_{i+1}, z_{i+1}$) is equal to the surface MSE.

To demonstrate that the non-monotonic warming is independent of the vertical coordinate, the results are also presented in height coordinates (Fig.~\ref{fig:fig-a1}). These profiles are obtained by following the same calculation as above except stepping in uniform intervals $\Delta z=100$ m. The pressure $p_{i+1}$ at height $z_{i+1}$ is calculated using hydrostatic balance.

\begin{figure}[htbp]
 \centering
 \includegraphics[width=\textwidth]{fig-a1.png}
 \caption{The moist adiabatic warming response to a 4 K surface warming in pressure coordinates. (a) Vertical profiles of the temperature response ($\Delta T$) as a function of pressure for surface temperatures ($T_s$) 280, 290, 300, 310, and 320 K. (b) The warming ($\Delta T$) at 5~km, 10~km, 15~km, and 20~km as a function of $T_s$. The non-monotonic behavior seen in height coordinates (Fig.~\ref{fig:fig-1}c) is also evident in pressure coordinates.}\label{fig:fig-a1}
\end{figure}


\begin{table}[htbp]
\caption{Thermodynamic constants used in the calculation of moist adiabatic profiles.}\label{tab:tableA1}
\begin{center}
\begin{tabular}{llcl}
\topline
Symbol & Description & Value & Units\\
\midline
$g$ & Acceleration due to gravity & 9.81 & m s$^{-2}$ \\
$c_p$ & Specific heat of dry air & 1005.7 & J kg$^{-1}$ K$^{-1}$ \\
$R_d$ & Gas constant for dry air & 287.05 & J kg$^{-1}$ K$^{-1}$ \\
$R_v$ & Gas constant for water vapor & 461.5 & J kg$^{-1}$ K$^{-1}$ \\
$\epsilon$ & Ratio of gas constants ($R_d/R_v$) & 0.622 & dimensionless \\
$p_s$ & Surface pressure & 1000 & hPa \\
$L_v$ & Latent heat of vaporization & $2.501 \times 10^6$ & J kg$^{-1}$ \\
\botline
\end{tabular}
\end{center}
\end{table}

\appendix[B] 
\appendixtitle{Calculation of Moist Adiabatic Profiles}
We assess how latent heat of fusion influences the non-monotonicity of moist adiabatic warming. We follow the IFS Cycle~40 approximations as summarized by \cite{flannaghan2014}. The fraction of liquid water $\alpha$ varies with $T$ as follows:
\begin{equation}
\alpha(T)=
\begin{cases}
0, & T \le T_{\mathrm{ice}},\\
\left(\dfrac{T-T_{\mathrm{ice}}}{T_0-T_{\mathrm{ice}}}\right)^2 & T_{\mathrm{ice}}<T<T_0,\\
1 & T \ge T_0,
\end{cases}
\end{equation}
where $T_{\mathrm{ice}}=253.15$ K and $T_0=273.15$ K. Thus all condensate is ice below 253.15 K, all condensate is liquid above 273.15 K, and a quadratic transition occurs in between.

The saturation vapor pressure $e_s$ is the weighted average over liquid ($e_\ell$) and ice ($e_i$):
\begin{equation}
e_s=\alpha e_{\ell}+(1-\alpha)e_i
\end{equation}
The saturation vapor pressure over liquid and ice is:
\begin{equation}
e_{\ell,i}(T) = a_1 \exp \left( a_3 \,\frac{T - T_0}{\,T - a_4\,} \right)
\label{eq:es_general}
\end{equation}
where over liquid $a_1=611.21$ Pa, $a_3=17.502$, $a_4=32.19$ K \citep{buck1981} and over ice $a_1=611.21$ Pa, $a_3=22.587$, $a_4=-0.7$ K \citep{alduchov1996}.

The effective latent heat of vaporization $L_e(T)$ includes both condensation and fusion:
\begin{equation}
L_e(T) = L_v + (1-\alpha) L_f
\end{equation}
where $L_f = 0.334 \times 10^6$~J~kg$^{-1}$ is the latent heat of fusion.

Moist adiabats are obtained by solving for $T$ that conserves moist static energy with the effective latent heat $L_e$:
\begin{equation}
    h = c_{pd}T + gz + L_e q_s
\end{equation}

The vertical profiles of warming $\Delta T$ and the warming at fixed pressure levels versus surface temperature exhibit similar non-monotonic behavior to the case without fusion (compare Fig.~\ref{fig:fig-1} and \ref{fig:fig-b1}). Latent heat of fusion introduces a secondary local maximum in the warming in the mid troposphere (500~hPa) due to the additional energy release from fusion. Since fusion represents a secondary effect and complicates analytical treatment, we neglect it for the rest of the analysis.

\begin{figure}[htbp]
 \centering
 \includegraphics[width=\textwidth]{fig-b1.png}
 \caption{The moist adiabatic warming response to a 4 K surface warming. (a) Vertical profiles of the temperature response ($\Delta T$) as a function of pressure for surface temperatures ($T_s$) of 280, 290, 300, 310, and 320 K. (b) The warming ($\Delta T$) at fixed pressure levels of 500, 400, 300, and 200 hPa as a function of $T_s$. The non-monotonic behavior seen without fusion (Fig.~\ref{fig:fig-1}c) is also evident in the presence of fusion.}\label{fig:fig-b1}
\end{figure}

\clearpage

%%%%%%%%%%%%%%%%%%%%%%%%%%%%%%%%%%%%%%%%%%%%%%%%%%%%%%%%%%%%%%%%%%%%%
% REFERENCES
%%%%%%%%%%%%%%%%%%%%%%%%%%%%%%%%%%%%%%%%%%%%%%%%%%%%%%%%%%%%%%%%%%%%%
% This shows how to enter the commands for making a bibliography using
% BibTeX. It uses references.bib and the ametsocV6.bst file for the style.

\bibliographystyle{ametsocV6}
\bibliography{references}


\end{document}
%%%%%%%%%%%%%%%%%%%%%%%%%%%%%%%%%%%%%%%%%%%%%%%%%%%%%%%%%%%%%%%%%%%%%
% END OF AMSPAPERV6.1.TEX
%%%%%%%%%%%%%%%%%%%%%%%%%%%%%%%%%%%%%%%%%%%%%%%%%%%%%%%%%%%%%%%%%%%%%